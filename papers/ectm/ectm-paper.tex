%% This is file `prletters-template.tex',
%% 
%% Copyright 2013 Elsevier Ltd
%% 
%% This file is part of the 'Elsarticle Bundle'.
%% ---------------------------------------------
%% 
%% It may be distributed under the conditions of the LaTeX Project Public
%% License, either version 1.2 of this license or (at your option) any
%% later version.  The latest version of this license is in
%%    http://www.latex-project.org/lppl.txt
%% and version 1.2 or later is part of all distributions of LaTeX
%% version 1999/12/01 or later.
%% 
%% The list of all files belonging to the 'Elsarticle Bundle' is
%% given in the file `manifest.txt'.
%% 
%% Template article for Elsevier's document class `elsarticle'
%% with harvard style bibliographic references
%%
%% $Id: prletters-template-with-authorship.tex 69 2013-07-15 10:15:25Z rishi $
%%
%% This template has no review option
%% 
%% Use the options `twocolumn,final' to obtain the final layout
\documentclass[times,twocolumn,final,authoryear]{elsarticle}

%% Stylefile to load PR Letters template
\usepackage{prletters}
\usepackage{framed,multirow}

%% The amssymb package provides various useful mathematical symbols
\usepackage{amssymb}
\usepackage{latexsym}

% Following three lines are needed for this document.
% If you are not loading colors or url, then these are
% not required.
\usepackage{url}
\usepackage{xcolor}
\definecolor{newcolor}{rgb}{.8,.349,.1}


% ----------------------------------------------------------
%                      CUSTOM COMMANDS
% ----------------------------------------------------------

\usepackage{amsmath}
\usepackage{todonotes}

\newcommand \bfshort[1] { {\todo[color=green!40]{#1 -- Bryan}} }

\newcommand \wdoc { { \mathbf{w}_d } }
\newcommand \xdoc { { \mathbf{x}_d } }

\newcommand \thd   { { \mathbf{\theta}_d } }
\newcommand \thdk  { { \theta_{dk} } }
\newcommand \vv[1] { { \mathbf{#1} } }
\newcommand \zdnk  { { z_{dnk} } }
\newcommand \wdnt  { { w_{dnt} } }

\newcommand \nor[2] { { \mathcal{N}\left(#1, #2\right) } }

% ----------------------------------------------------------
%                      / CUSTOM COMMANDS
% ----------------------------------------------------------


\journal{Pattern Recognition Letters}

\begin{document}

\thispagestyle{empty}
                                                             
\begin{table*}[!th]

\begin{minipage}{.9\textwidth}
\baselineskip12pt
\ifpreprint
  \vspace*{1pc}
\else
  \vspace*{-6pc}
\fi

\noindent {\LARGE\itshape Pattern Recognition Letters}
\vskip6pt

\noindent {\Large\bfseries Authorship Confirmation}

\vskip1pc


{\bf Please save a copy of this file, complete and upload as the 
``Confirmation of Authorship'' file.}

\vskip1pc

As corresponding author 
I, Bryan Feeneey, 
hereby confirm on behalf of all authors that:

\vskip1pc

\begin{enumerate}
\itemsep=3pt
\item This manuscript, or a large part of it, \underline {has not been
published,  was not, and is not being submitted to} any other journal. 

\item If \underline {presented} at or \underline {submitted} to or
\underline  {published }at a conference(s), the conference(s) is (are)
identified and  substantial \underline {justification for
re-publication} is presented  below. A \underline {copy of
conference paper(s) }is(are) uploaded with the  manuscript.

\item If the manuscript appears as a preprint anywhere on the web, e.g.
arXiv,  etc., it is identified below. The \underline {preprint should
include a  statement that the paper is under consideration at Pattern
Recognition  Letters}.

\item All text and graphics, except for those marked with sources, are
\underline  {original works} of the authors, and all necessary
permissions for  publication were secured prior to submission of the
manuscript.

\item All authors each made a significant contribution to the research
reported  and have \underline {read} and \underline {approved} the
submitted  manuscript. 
\end{enumerate}

Signature\underline{\hphantom{\hspace*{7cm}}} Date\underline{\hphantom{\hspace*{4cm}}} 
\vskip1pc

\rule{\textwidth}{2pt}
\vskip1pc

{\bf List any pre-prints:}
\vskip5pc


\rule{\textwidth}{2pt}
\vskip1pc

{\bf Relevant Conference publication(s) (submitted, accepted, or
published):}
\vskip5pc



{\bf Justification for re-publication:}

\end{minipage}
\end{table*}

\clearpage
\thispagestyle{empty}
\ifpreprint
  \vspace*{-1pc}
\fi

\begin{table*}[!th]
\ifpreprint\else\vspace*{-5pc}\fi

\section*{Graphical Abstract (Optional)}
To create your abstract, please type over the instructions in the
template box below.  Fonts or abstract dimensions should not be changed
or altered. 

\vskip1pc
\fbox{
\begin{tabular}{p{.4\textwidth}p{.5\textwidth}}
\bf Type the title of your article here  \\
Author's names here \\[1pc]
\includegraphics[width=.3\textwidth]{top-elslogo-fm1.pdf}
& 
This is the dummy text for graphical abstract.
This is the dummy text for graphical abstract.
This is the dummy text for graphical abstract.
This is the dummy text for graphical abstract.
This is the dummy text for graphical abstract.
This is the dummy text for graphical abstract.
This is the dummy text for graphical abstract.
This is the dummy text for graphical abstract.
This is the dummy text for graphical abstract.
This is the dummy text for graphical abstract.
This is the dummy text for graphical abstract.
This is the dummy text for graphical abstract.
This is the dummy text for graphical abstract.
This is the dummy text for graphical abstract.
This is the dummy text for graphical abstract.
This is the dummy text for graphical abstract.
This is the dummy text for graphical abstract.
This is the dummy text for graphical abstract.
This is the dummy text for graphical abstract.
This is the dummy text for graphical abstract.
This is the dummy text for graphical abstract.
This is the dummy text for graphical abstract.
This is the dummy text for graphical abstract.
This is the dummy text for graphical abstract.
%}\\
\end{tabular}
}

\end{table*}

\clearpage
\thispagestyle{empty}

\ifpreprint
  \vspace*{-1pc}
\else
%  \vspace*{-6pc}
\fi

\begin{table*}[!t]
\ifpreprint\else\vspace*{-15pc}\fi

\section*{Research Highlights (Required)}

\vskip1pc

\fboxsep=6pt
\fbox{
\begin{minipage}{.95\textwidth}
It should be short collection of bullet points that convey the core
findings of the article. It should  include 3 to 5 bullet points
(maximum 85 characters, including spaces, per bullet point.)  
\vskip1pc
\begin{itemize}
\item A method to predict topics for microtexts which are otherwise overparameterised
\item A novel use of the Bohning bound in conjunction with correlated topic models
\item A novel use of a low-rank matrix-variate (LRMV) prior over features and topics
\item An novel and efficient inference scheme for the LRMV prior reducing complexity from $O(P^3Q^3)$ to $O(P^3 + Q^3)$
\begin{itemize}
	\item This can be extended to other multi-task methods such as co-kriging GPs
\end{itemize}
\item A demonstration on two real-world problems
\begin{itemize}
	\item Better prediction of tags for images using features learned from deep nets
	\item Better prediction of hashtags for tweets using authorship and time
\end{itemize}

\end{itemize}
\vskip1pc
\end{minipage}
}

\end{table*}

\clearpage


\ifpreprint
  \setcounter{page}{1}
\else
  \setcounter{page}{1}
\fi

\begin{frontmatter}

\title{Multi-Task Prediction of Topic Admixtures for Micro-Documents}

\author[1]{Bryan \snm{Feeney}\corref{cor1}} 
\cortext[cor1]{Corresponding author: 
  Tel.: +44-77-7096-4588}
\ead{bryan.feeney@gmail.com}
\author[2]{Cédric \snm{Archambeau}}
\author[1]{Ricardo \snm{Silva}}

\address[1]{University College London, Gower Street, London WC1E 6BT, United Kingdom}
\address[2]{Amazon Research, Krausenstr. 38, 10117 Berlin, Germany}

\received{1 May 2013}
\finalform{10 May 2013}
\accepted{13 May 2013}
\availableonline{15 May 2013}
\communicated{S. Sarkar}


\begin{abstract}
Topic models are well established as a means of creating low-dimensional representations of text. Such text documents are frequently associated with other information such as authorship, date or richer information sources such as images. Use of this information can improve the assignments of topic mixtures to documents: this is particularly useful for very small documents, such as tweets and image captions. In this paper we present a conditional topic model which uses associated information to help fit topics. By means of a matrix-variate prior on weights we capture correlations between both topics and features, which is essential to make accurate predictions when the number of topics is very high. We provide a fast, efficient inference scheme, and demonstrate the model's effectiveness on two corpora: a selection of  tweets from late 2016, and a selection of images with tags.
\end{abstract}

\begin{keyword}
\MSC 41A05\sep 41A10\sep 65D05\sep 65D17
\KWD Keyword1\sep Keyword2\sep Keyword3

%% MSC codes here, in the form: \MSC code \sep code
%% or \MSC[2008] code \sep code (2000 is the default)
\end{keyword}

\end{frontmatter}

%\linenumbers

%% main text


\subsection{Introduction}
Bayesian multinomial admixture models such as Latent Dirichlet Allocation\cite{BleiNgJordan2003} are a popular way of creating low-dimensional representations of text. \todo{Why}

While LDA itself is concerned solely with the contents of documents, documents typically have other information available to them, and numerous models have been proposed to exploit such extra information, including date of publication\citep{Blei2006a}\citep{Wang2006}, geography\citep{Eisenstein2010}, and authorship\citep{RosenZvi2004}\citep{MacCallum2007} as well as images\citep{Blei2003} and paired documents (e.g. in other languages\citep{Zhao2001}).

Typically the field has been divided into two classes\citep{Mimno2008}: ``upstream" models model the conditional distribution of a document's words given its observed features $p(\wdoc | \xdoc)$, where $\wdoc \in \mathbb{R}^T$ is the bag-of-words representation of text, and $\xdoc \in \mathbb{R}^F$ is a vector of features. By comparison ``downstream" models model documents and observed features concurrently $p(\wdoc, \xdoc$). This latter class are also sometimes known as ``multi-modal" topic models.

While the use of ad-hoc methods has had some success, there have been efforts to create generic models that can be applied to a variety of datasets. In the case of ``downstream" models early methods generated multiple observations from a single shared topic mixture\citep{Blei2003}\citep{Erosheva2004}\citep{Yakhnenko2009}, however often the number of topics needed to model certain ``modalities" or classes of observation varies, leading to newer models extending the correlated topic-model\citep{Blei2006} to create models where separate but correlated topic-mixtures for each modality\citep{Salomatin2009}\citep{Virtanen2012a} are generated by sampling a long topic-strength vector from a Gaussian and then partitioning it by modality.

In the case of upstream models, there has been much less work. The most significant model was the Dirichlet Multinomial Regression (DMR) \citep{Mimno2008} model which generated a topic-strength vector as a simple linear function of the associated features an a topic-specific weight vector. This was extended to use kernel methods in \citep{Hennig2012}

A disadvantage of the DMR is that it involves independently and separately the different weight vectors $\vv{w}_k$ for each topic -- and there may be over a hundred topics to learn.



\subsection{Entering text}
\textcolor{newcolor}{\bf Please note that Full Length Papers can have 7
pages (plus one page after revision) and Special Issue Papers can have
10 pages (plus one page after revision). The only exception is the
review article that is submitted to a Special Issue. These limits
include all materials e.g. narrative, figures, tables, references,
etc.}

\section{The first page}
Avoid using abbreviations in the title. Next, list all authors with
their first names or initials and surnames (in that order). Indicate
the author for correspondence (see elsarticle documentation).

Present addresses can be inserted as footnotes. After having listed all
authors' names, you should list their respective affiliations. Link
authors and affiliations using superscript lower case letters.

\subsection{The Abstract}
An Abstract is required for every paper; it should succinctly summarize
the reason for the work, the main findings, and the conclusions of the
study. The abstract should be no longer than 200 words. Do not include
artwork, tables, elaborate equations or references to other parts of
the paper or to the reference listing at the end. ``Comment'' papers
are exceptions, where the commented paper should be referenced in full
in the Abstract.

The reason is that the Abstract should be understandable in itself to
be suitable for storage in textual information retrieval systems.

\textit{Example of an abstract: A biometric sample collected in an
uncontrolled outdoor environment varies significantly from its
indoor version. Sample variations due to outdoor environmental
conditions degrade the performance of biometric systems that
otherwise perform well with indoor samples. In this study, we
quantitatively evaluate such performance degradation in the case
of a face and a voice biometric system. We also investigate how
elementary combination schemes involving min-max or z
normalization followed by the sum or max fusion rule can
improve performance of the multi-biometric system. We use
commercial biometric systems to collect face and voice samples
from the same subjects in an environment that closely mimics the
operational scenario. This realistic evaluation on a dataset of
116 subjects shows that the system performance degrades in
outdoor scenarios but by multimodal score fusion the
performance is enhanced by 20\%. We also find that max rule
fusion performs better than sum rule fusion on this dataset. More
interestingly, we see that by using multiple samples of the same
biometric modality, the performance of a unimodal system can
approach that of a multimodal system.}

\section{The main text}

Please divide your article into (numbered) sections (You can find the
information about the sections at
\url{http://www.elsevier.com/wps/find/journaldescription.cws_home/505619/authorinstructions}).
Ensure that all tables, figures and schemes are cited in the text in
numerical order. Trade names should have an initial capital letter, and
trademark protection should be acknowledged in the standard fashion,
using the superscripted characters for trademarks and registered
trademarks respectively. All measurements and data should be given in
SI units where possible, or other internationally accepted units.
Abbreviations should be used consistently throughout the text, and all
nonstandard abbreviations should be defined on first usage
\citep{vanderGeeretal2000}.

\begin{table*}[!t]
\caption{\label{tab1}Summary of different works pertaining to face and
speech fusion}
\centering
\begin{tabular}{|p{2.25cm}|p{2cm}|l|p{4cm}|p{3cm}|p{2cm}|}
\hline
Study & Algorithm used & DB Size & Covariates of interest & 
Top individual performance & Fusion\newline Performance\\
\hline
UK-BWG
(Mansfield et al.,
2001) &
Face, voice:\newline
Commercial & 200 & Time: 1--2 month\newline
separation (indoor) & 
TAR$^*$ at 1\% FAR$^{\#}$\newline
Face: 96.5\%\newline
Voice: 96\%
& --\\
\hline
Brunelli
(Brunelli and
Falavigna, 1995) & 
Face:\newline
Hierarchical\newline
correlation\newline
Voice:\newline
MFCC & 
87 & 
Time: 3 sessions, time\newline
unknown (indoor) & 
Face:\newline
TAR = 92\% at\newline
4.5\% FAR\newline
Voice:\newline
TAR = 63\% at\newline
15\% FAR
&
TAR =98.5\%\newline
at 0.5\% FAR\\
\hline
Jain
(Jain et al., 1999)
&
Face:\newline
Eigenface\newline
Voice:\newline
Cepstrum\newline
Coeff. Based
&
50
&
Time: Two weeks (indoor)
&
TAR at 1\% FAR\newline
Face: 43\%\newline
Voice: 96.5\%\newline
Fingerprint: 96\%
& 
Face $+$ Voice $+$\newline
Fingerprint $=$\newline
98.5\%\\
\hline
Sanderson
(Sanderson and
Paliwal, 2002)
&
Face: PCA\newline
Voice: MFCC &
43 
& Time: 3 sessions (indoor)\newline
Noise addition to voice & 
Equal Error Rate\newline
Face: 10\%\newline
Voice: 12.41\%
&
Equal Error\newline
Rate 2.86\% \\
\hline
Proposed study & 
Face, voice:\newline
Commercial & 116 &
Location: Indoor and\newline
Outdoor (same day)\newline
Noise addition to eye\newline 
coordinates 
&
TARs at 1\% FAR\newline
Indoor-Outdoor\newline
Face: 80\%\newline
Voice: 67.5\%
&
TAR = 98\%\newline
at 1\% FAR\\
\hline
\multicolumn{6}{@{}l}{$^*$TAR--True Acceptance Rate\qquad 
$^{\#}$ FAR--False Acceptance Rate}
\end{tabular}
\end{table*}


\subsection{Tables, figures and schemes}
Graphics and tables may be positioned as they should appear in the
final manuscript. Figures, Schemes, and Tables should be numbered.
Structures in schemes should also be numbered consecutively, for ease
of discussion and reference in the text. \textcolor{newcolor}{\bf
Figures should be maximum half a page size.}
All numbers and letters in figures and diagrams should be at least of
the same font size as that of the figure caption.

Depending on the
amount of detail, you can choose to display artwork in one column (20
pica wide) or across the page (42 pica wide). Scale your artwork in
your graphics program before incorporating it in your text. If the
artwork turns out to be too large or too small, resize it again in your
graphics program and re-import it. The text should not run along the
sides of any figure. This is an example for citation \citep{StrunkWhite1979}.

\begin{figure}[!t]
\centering
\includegraphics[scale=.5]{prletfig01}
\caption{Studio setup for capturing face images indoor. Three light
sources L1, L2, L3 were used in conjunction with normal office lights.}
\end{figure}

You might find positioning your artwork within the text difficult
anyway. In that case you may choose to place all artwork at the end of
the text and insert a marker in the text at the desired place. In any
case, please keep in mind that the placement of artwork may vary
somewhat in relation to the page lay-out \citep{MettamAdams1999}.

This can easily be achieved using \verb+endfloat.sty+ package. Please
refer the following documentation to use this package.
\makeatletter
\if@twocolumn
\begin{verbatim}
  http://mirrors.ctan.org/macros/latex/contrib/
  endfloat/endfloat.pdf
\end{verbatim}
\else
\begin{verbatim}
  http://mirrors.ctan.org/macros/latex/contrib/endfloat/endfloat.pdf
\end{verbatim}
\fi
\makeatother

\textcolor{newcolor}{\bf You should insert a caption for the figures
below the figures and for the tables the caption should be above the
tables.} 

Please remember that we will always also need highresolution versions
of your artwork for printing, submitted as separate files in standard
format (i.e. TIFF or EPS), not included in the text document. Before
preparing your artwork, please take a look at our Web page:
\url{http://www.elsevier.com/locate/authorartwork}.


\subsection{Lists}

For tabular summations that do not deserve to be presented as
a table, lists are often used. Lists may be either numbered or
bulleted. Below you see examples of both.
\begin{enumerate}
\item The first entry in this list
\item The second entry
\begin{enumerate}
\item A subentry
\end{enumerate}
\item The last entry
\end{enumerate}
\begin{itemize}
\item A bulleted list item
\item Another one
\end{itemize}

\subsection{Equations}
Conventionally, in mathematical equations, variables and
anything that represents a value appear in italics.
All equations should be numbered for easy referencing. The number
should appear at the right margin.
\begin{equation}
S_{\rm pg}'=\frac{S_{\rm pg}-\min(S_{\rm pG})}
 {\max(S_{\rm pG}-\min(S_{\rm pG})}
\end{equation}
In mathematical expressions in running text ``/'' should be used for
division (not a horizontal line). 

\section*{Acknowledgments}
Acknowledgments should be inserted at the end of the paper, before the
references, not as a footnote to the title. Use the unnumbered
Acknowledgements Head style for the Acknowledgments heading.

\section*{References}

Please ensure that every reference cited in the text is also present in
the reference list (and vice versa).

\section*{\itshape Reference style}

Text: All citations in the text should refer to:
\begin{enumerate}
\item Single author: the author's name (without initials, unless there
is ambiguity) and the year of publication;
\item Two authors: both authors' names and the year of publication;
\item Three or more authors: first author's name followed by `et al.'
and the year of publication.
\end{enumerate}
Citations may be made directly (or parenthetically). Groups of
references should be listed first alphabetically, then chronologically.

\bibliographystyle{model2-names}
\bibliography{/Users/bryanfeeney/Documents/library.bib}

\section*{Supplementary Material}

Supplementary material that may be helpful in the review process should
be prepared and provided as a separate electronic file. That file can
then be transformed into PDF format and submitted along with the
manuscript and graphic files to the appropriate editorial office.

\end{document}

%%

\documentclass[10pt,fleqn]{article}

\usepackage[english]{babel}
\usepackage[utf8x]{inputenc}
\usepackage{enumerate}
\usepackage{amsmath}
\usepackage{amssymb}
\usepackage{amsfonts} 
\usepackage{mathtools}
\usepackage{graphicx}
\usepackage{bm}
\usepackage[usenames,dvipsnames]{color}
\usepackage{todonotes}
\usepackage{dsfont}
\usepackage{hyperref}
\hypersetup{
    colorlinks,
    citecolor=black,
    filecolor=black,
    linkcolor=black,
    urlcolor=black
}
\usepackage{algorithm}
\usepackage{algorithmic}
\usepackage{appendix}
\usepackage{subcaption}
\usepackage{fancyvrb}
\usepackage{subfigure}
\usepackage{graphicx,xcolor}
\usepackage{pifont,mdframed}
\usepackage{tikz}
\usepackage{bm}
\usetikzlibrary{fit,positioning}


%
% Macros
%
\newcommand \cashort[1] { {\todo[color=yello]{#1 -- Cedric}} }
\newcommand \calong[1]  { { \todo[inline,color=yellow]{#1 -- Cedric} } }
\newcommand \gbshort[1] { {\todo[color=cyan!40]{#1 -- Guillaume}} }
\newcommand \gblong[1]  { { \todo[inline, color=cyan!40]{#1 -- Guillaume} } }
\newcommand \mgshort[1] { {\todo{#1 -- Mark}} }
\newcommand \mglong[1]  { { \todo[inline]{#1 -- Mark} } }
\newcommand \bfshort[1] { {\todo[color=green!40]{#1 -- Bryan}} }
\newcommand \bflong[1]  { { \todo[inline,color=green!40]{#1 -- Bryan} } }


% Adds a plus const to the end of a math expression
\def \pcst{+\text{const}}

% A fancy version for capital R
\def \Rcal{\mathcal{R}}

% A fancy version for r
\def \rcal{\mathbf{r}}

% Loss function / log likelihood as appropriate
\def \L{\mathcal{L}}

% KL divergence [Math Mode]
\newcommand{\kl}[2] {
	\text{KL}\left[#1||#2\right]
}

\newcommand \vecf[1] {
    \text{vec}\left(#1\right)
}

\newcommand \ent[1] {
    \text{H} \left[ #1 \right]
}

\newcommand \mut[2] {
    \text{I} \left[ #1 ; #2 \right]
}

\newcommand \dvi[2] {
    \text{D}_\text{VI} \left[ #1; #2 \right]
}

% Starts an expected value expresses [Math Mode]
\newcommand{\starte}[1] {%
	\mathbb{E}_{#1}\left[
}

% Ends an expected value expression [Math Mode]
\def \ende{\right]}

% Starts an varianc expresses [Math Mode]
\newcommand{\startv}[1] {%
	\mathbb{V}\text{ar}_{#1}\left[
}

% Ends an variance expression [Math Mode]
\def \endv{\right]}

%\newcommand \ex[2] {
%    \bigl\langle #1 \bigr\rangle_{#2}
%}
\newcommand \ex[2] {
    \mathbb{E}_{ { #2 } }\left[ #1 \right]
}
\newcommand \var[2] {
    \mathbb{V}ar_{ { #2 } }\left[ #1 \right]
}

\newcommand \halve[1] {
	\frac{#1}{2}
}

\newcommand \half {
    \halve{1}
}

\newcommand \tr { \text{tr} } 

\newcommand \T { ^\top } 

\newcommand \fixme[1] {
    {\color{red} FIXME: #1}
}

\newcommand \vv[1] { \bm #1 }

\newcommand{\mbeq}{\overset{!}{=}}

\newcommand \diag[1] { \text{diag} \left( {#1} \right) }
\newcommand \diagonal[1] { \text{diagonal} \left( {#1} \right) }

\newcommand \Ed {{ \vv{\xi}_d}}
\newcommand \Edj {{\xi_{dj}}}
\newcommand \Edk {{\xi_{dk}}}
\newcommand \AEdj {{\Lambda(\xi_{dj})}}
\newcommand \AEdk {{\Lambda(\xi_{dk})}}
\newcommand \AEd  {{ \bm{\Lambda}(\bm{\xi}_d) }}

\newcommand \Axi { { \Lambda_{\xi} } }
\newcommand \bxi { { \vv{b}_{\xi} } }
\newcommand \cxi { { c_{\xi} } }


\newcommand \wdoc      { { \vv{w}_d } }
\newcommand \wdt[0]  { { w_{dt} } }
\newcommand \wdn[0]  { { \vv{w}_{dn} } }
\newcommand \wdnt[0]  { { w_{dnt} } }
\newcommand \wdd[0]   { { \vv w_{d} } }
\newcommand \zd[0]   { { \vv z_{d} } }
\newcommand \zdn[0]  { { \vv{z}_{dn} } }
\newcommand \zdnk[0] { { z_{dnk} } }
\newcommand \zdk[0]  { { z_{dk} } }
\newcommand \thd[0]  { { \vv \theta_d } }
\newcommand \thdk[0] { { \theta_{dk} } }
\newcommand \thdj[0] { { \theta_{dj} } }
\newcommand \epow[1] { { e^{#1} } }
\newcommand \pkt     { { \phi_{kt}  } }
\newcommand \pk      { { \vv \phi_k } }
\newcommand \lmd     { { \vv \lambda_d } }
\newcommand \lmdk    { { \lambda_{dk} } }
\newcommand \xd      { { \vv x_d } }
\newcommand \atxd     { A ^\top \bm x_d}
\newcommand \axd     { A\bm x_d}
\newcommand \tsq      { { \tau^2 } }
\newcommand \ssq      { { \sigma^2 } }
\newcommand \tmsq     { { \tau^{-2} } }
\newcommand \asq      { { \alpha^2 } }
\newcommand \amsq     { { \alpha^{-2} } }
\newcommand \sgsq     { { \sigma^2 } }
\newcommand \xvec     { { \vv{x} } }
\newcommand \omk      { { \bm \omega _k } }
\newcommand \omkt     { { \omega_{kt} } }
\newcommand \oma     { { \Omega_A } }
\newcommand \gdn      { { \vv{\gamma}_{dn} } }
\newcommand \gdnk     { { \gamma_{dnk} } }
\newcommand \gdk      { { \gamma_{dk} } }
\newcommand \isigt   { { \Sigma^{-1}_{\bm \theta} } }




\newcommand \halfSig { \frac{1}{2\sigma^2} }

\newcommand \nor[2]   { \mathcal{N} \left( {#1}, {#2} \right) }
\newcommand \nord[3]   { \mathcal{N}_{#1} \left( {#2}, {#3} \right) }
\newcommand \mnor[3]  { \mathcal{N} \left(#1, #2, #3\right) }
\newcommand \norp[3]  { \mathcal{N} \left(#1; #2, #3\right) }
\newcommand \mnorp[4] { \mathcal{N} \left(#1; #2, #3, #4\right) }
\newcommand \mul[1]   { \mathcal{M} \left( {#1} \right) }
\newcommand \muln[2]  { \mathcal{M} \left( {#1},{#2} \right) }
\newcommand \dir[1]   { \mathcal{D} \left( {#1} \right) }
\newcommand \pois[1]  { \mathcal{P} \left( {#1} \right) }
\newcommand \gp[2]    { \mathcal{GP} \left( {#1}, #2 \right) }
\newcommand \dir[1]   { \mathcal{D} \left( {#1} \right) }
\newcommand \gam[2]   { \mathcal{G} \left( {#1}, {#2} \right) }
\newcommand \beta[1]  { \mathcal{B}eta \left( {#1}, {#2} \right) }

\newcommand \lne[1]  { { \ln \left( 1 + e^{ #1 } \right) } }
\newcommand \Tr[1]   { \tr \left(  {#1}  \right) }

\newcommand \roud  { \vv{\rho}_{d}  }
\newcommand \rodk { \rho_{dk} }

\newcommand \exA[1]  { \ex{#1}{q(A)} }
\newcommand \exV[1]  { \ex{#1}{q(V)} }
\newcommand \exT[1]  { \ex{#1}{q(\Theta)} }
\newcommand \extd[1] { \ex{#1}{q(\thd)} }
\newcommand \exTV[1] { \ex{#1}{q(\Theta)q(V)} }

\newcommand \Real[0]  { { \mathbb{R} } }
\newcommand \VReal[1] { { \mathbb{R}^{#1} } }
\newcommand \MReal[2] { { \mathbb{R}^{#1 \times #2} } }
\newcommand \Nat[0]  { { \mathbb{N} } }
\newcommand \VNat[1] { { \mathbb{N}^{#1} } }
\newcommand \MNat[2] { { \mathbb{N}^{#1 \times #2} } }

\newcommand \inv[1] { {#1}^{-1} }
\newcommand \invb[1] { \inv{\left( #1 \right)} }

\newcommand \cn { \textsuperscript{\texttt{[{\color{blue}Citation Needed}]}} }

\newcommand \const { { \text{c} } }

\providecommand \floor [1] { \left \lfloor #1 \right \rfloor }
\providecommand \ceil [1] { \left \lceil #1 \right \rceil }


\newcommand \vt[2] { { #1^{(#2)} } }

\newcommand \hashtag[1] { { \ttfamily \##1 } }

\newcommand \mvy  { \vv{m}_{\vv{y}} }
\newcommand \sigvy { { S_Y } }

\newcommand \mmy  { M_Y      }
\newcommand \md   { \vv{m}_d }
\newcommand \phin { \vv{\phi}_n }
\newcommand \isigma { { \inv{\Sigma} } }

\newcommand \sigv     { { \Sigma_V } }
\newcommand \isigv     { { \Sigma^{-1}_V } }

\newcommand \sigy { { \Sigma_Y } }
\newcommand \isigy { { \Sigma_{-1}_Y } }


\newcommand \omy  { { \Omega_Y } }
\newcommand \iomy { { \inv{\Omega_Y} } }

\newcommand \siga     { { \Sigma_A } }
\newcommand \isiga     { { \Sigma^{-1}_A } }
\newcommand \diagv { { \diag{\nu_1,\ldots,\nu_P} } }

\newcommand \ma { \vv{m}_a }
\newcommand \my { \vv{m}_y }

\newcommand \VoU { V \otimes U }

\newcommand \one { \mathbb{1} }
%\newcommand \one  {{  \mathds{1} }}

\newcommand \lse { \text{lse} }
%\newcommand \lse[0] { \mathrm{lse} }

% Conditional independence 
\def\ci{\perp\!\!\!\perp} % from Wikipedia



% ------ For the eval section

% Multinomial PDF [Math Mode]
% params: 1 - the variable
%         2 - the value
%         3 - the state indicator (e.g. k for a distro with K values)
%         4 - any additional subscript
\newcommand{\mpdf}[4] {
	\prod_{#3} {#1}_{{#4} {#3}} ^ {#2}
}

% Dirichlet PDF [Math Mode]
% params: 1 - the variable
%         2 - the hyper-parameter
%         3 - the state indicator (e.g. k for a distro with K values)
%         4 - any additional subscript
\newcommand{\dpdf}[4] {
	\frac{1}{B({#2})} \prod_{#3} {#1}_{{#4} {#3}} ^ {({#2}_{#3} - 1)}
}

% To simplify the sampling equations, this is indicates
% that the given value has had datapoint "m" stripped out
%
\newcommand{\lm}[1] {
	#1^{\setminus m}
}

\newcommand \model[0] {
    \mathcal{M}
}

\newcommand \perplexity[1] {
    \mathcal{P} \left( { #1 } \right)
}

\newcommand \WTrain {
    \mathcal{W}^{(t)}
}

\newcommand \WQuery {
    \mathcal{W}^{(q)}
}

\newcommand \oneover[1] {
    \frac{1}{ {#1} }
}

\newcommand \samp[1] {
    { #1 }^{(s)}
}

\newcommand \etd[0] {
    \vv{\eta}_d
}

\begin{document}


\newcommand \ed { { \vv{\xi}_d } }
\newcommand \bd { { \vv{b}_d   } }

\section{Appendix}

\subsection{The Variational Lower Bound}
To recap, our model posits that each for each document $d$ in our corpus, its topic score-vector $\thd \in \VReal{K}$ is a function of an observed vector of document features $\xd \in \VReal{F}$ and an unknown matrix $A$. This is converted to topic-distribution using the softmax function $\vv{\sigma}(\vv{\theta}) = \left\{ \frac{\exp(\theta_k)}{\sum_j \exp(\theta_j)} \right\}_{k=1}^K$. For each position in the document we draw a topic $\zdn$ according to this distribution, and then draw a word from the topic-specific word-distribution $\pk$

\begin{align}
\thd &\sim \nor{A\T\xd}{\sigma I_K} &
\zdn &\sim \muln{\vv{\sigma}(\thd)}{1} & 
\wdn &\sim \prod_k z_{dnk} \muln{\pk}{1}
\end{align}

The word distributions have a symmetric Dirichlet prior $\pk \sim \dir{\vv{\beta}}$. For $A \in \MReal{F}{K}$ we use a matrix-variate prior conditioned on a latent variable $Y \in \MReal{P}{Q}$ where $P << F$ and $Q << K$.

\begin{align}
A &\sim \mnor{UYV\T}{\sigma I_K}{\tau I_F} &
Y &\sim \mnor{0}{\alpha I_Q}{\beta I_P}
\end{align}

while $U \in \MReal{F}{P}$ and $V \in \MReal{K}{Q}$ are parameters with no distribution. Letting $n_d$ be the total number of words in document $d$, $\Theta = \left\{ \thd\T \right\}_{d=1}^D$, $\mathcal{Z} = \bigcup_{d=1}^D \{ z_{dn} | n \in \{1, \ldots, n_d \} \}$ and $\mathcal{W} = \bigcup_{d=1}^D \{ w_{dn} | n \in \{1, \ldots, n_d \} \}$ we approximate the true posterior $p(Y, A, \Theta, \mathcal{Z} | \mathcal{W})$ with the factorised approximate posterior

\fixme{How to allow for MAP estimation of $\pk$ in this framework}

\begin{align}
q(Y, A, \Theta, \mathcal{Z}) = q(Y)q(A)\prod_d q(\thd) \prod_n q(\zdn)
\end{align}
The posterior factors have the following distributions

\begin{equation}
\begin{split}
q(Y) &= \nor{\vecf{M_Y}}{S_y} &
q(A) &= \mnor{M_A}{R_A}{S_A} \\
q(\thd) &= \nor{\md}{\diag{\vv{s}_d}} &
q(\zdn) & = \muln{\gdn}{1} \\
\end{split}
\end{equation}


Using Jensen's inequality\fixme{citation} with the approximate bound, we obtain the following lower-bound on the log-evidence, $\ln p(W)$

\begin{equation}
\begin{split}
\ln p(W) 
&\geq \ex{\ln p(Y)}{q} + \ex{\ln p(A)}{q} \\
&+ \sum_d \ex{\ln p(\thd)}{q} + \sum_n \ex{\ln p(\zdn)}{q} + \ex{\ln p(\wdn)}{q} \\
&+ \ent{Y} + \ent{A} + \sum_d \ent {\thd} + \sum_n \ent{\zdn}
\end{split}
\end{equation}

The posterior distributions are

\begin{align}
\begin{split}
\ex{\ln p(Y)}{q} 
    & = -\halve{PQ} \ln 2\pi - \halve{PQ} \ln \alpha \beta - \\
    & - \half \Tr{ \invb{\alpha \beta} I_P M_Y M_Y\T } \\
    & - \half \Tr{\invb{\alpha \beta} I_{PQ} S_y }
\end{split} \\
\begin{split}
\ex{\ln p(A)}{q} 
    & = -\halve{KF} \ln 2\pi - \halve{KF} \ln \sigma - \halve{KF} \ln \tau \\
    & - \half \Tr{ \inv{\tau} I_F (M_A - U M_Y V\T) \inv{\sigma} I_K (M_A - U M_Y V\T)\T } \\
    & - \half \Tr{ \inv{\tau} I_F S_A}\Tr{\inv{\sigma} I_K R_A} \\
    & - \half \Tr{ U\T \inv{\tau}I_F U}\Tr{ V\T \inv{\sigma} I_K V)}
\end{split} \\
\begin{split}
\ex{\ln p(\Theta)}{q} 
    & = -\halve{DK} \ln 2\pi - \halve{DK} \ln \sigma  \\
    & - \half \sum_d \left(\md - M_A\T \xd\right)\T \inv{\sigma} I_K \left(\md - M_A\T \xd\right) \\
    & - \half \sum_d \Tr{ \inv{\sigma} I_K \diag{\vv{s}_d}} \\
    & - \halve{D} \Tr{ X\T X S_A}\Tr{\inv{\sigma} I_K R_A}
\end{split}\\
\begin{split}
\ex{\ln p(\mathcal{W})}{q} 
    & = \sum_d \sum_n \sum_k \sum_t \gdnk \wdnt \ln \phi_{kt}
\end{split}
\end{align}

Using the Bohning bound\cite{Bohning1988} we can lower bound the expected log-probability of the topic-assignments as

\begin{equation}
\begin{split}
\ex{\ln p(\mathcal{Z})}{q} 
     = & \sum_d \sum_n \sum_k \gdnk m_{dk} - \lse(\md) \\
    \geq & \sum_d \vv{\gamma}_{d\cdot}\T \md
      - \sum_d n_d \left( \half \md\T H \md  - \bd \T \md + c_d \right) \\
      & - \halve{n_d} \Tr{\diag{\vv{s}_d}H}
\end{split}
\end{equation}

where $\vv{\gamma}_{d\cdot} = \sum_n \gdn$ and $H$, $\vv{b}_d$ and $c_d$ are additional parameters introduced by the bound.

The entropies are given by

\begin{align}
\ent{Y} &= \halve{PQ} \ln 2 \pi e + \halve{P}\ln |R_Y| + \halve{Q} |S_Y| \\
\ent{Y} &= \halve{FK} \ln 2 \pi e + \halve{F}\ln |R_A| + \halve{K} |S_A| \\
\ent{\thd} &= \half{K} \ln 2 \pi e + \sum_k \ln s_{dk}
\ent{\zdn} &= -\sum_k \gdnk \ln \gdnk
\end{align}



\section{The Posterior Distributions}

\subsection{q(Y)}
Taking derivatives we obtain the solution

\begin{align}
-\invb{\alpha \beta} M_Y - U\T U M_Y V V\T + U\T M_A V \mbeq 0
\end{align}

This Sylvester equation can be solved by means of the transformation $\vecf{ABC} = (C\T \otimes A)\vecf{B}$. This leads to the solution

\begin{align}
\vecf{M_Y} = \invb{\invb{\alpha \beta} I_{PQ} + U\T U \otimes V\T V}\vecf{U\T M_A V} \label{eqn:dumb-y-soln}
\end{align}


Similar calculations show the posterior covariance to be 

\begin{align}
S_y = \invb{\invb{\alpha \beta} I_{PQ} + U\T U \otimes V\T V}
\end{align}

As the dimensions of $S_y \in \MReal{(P\times Q)}{(P \times Q)}$ are excessive, we use the following method to avoid the inverse when deriving the update for $M_Y$.

Given the eigen-decomposition $B = B_A S_B U_B\T$, where $S_A$ is the diagonal matrix of eigenvalues, we can write $\inv{B} = U_B \inv{S_B} U_B\T$. Using the result in \cite{Stegle2011} we can further write that for a matrix $A = \alpha I + B \otimes C$ its eigen-decomposition is given by the matrices

\begin{align}
A & = U_A S_A U_A\T & U_A & = U_C \otimes U_B & S_A = \alpha I + S_C \otimes S_B
\end{align}

where the eigen-decomposition of $B$ is $U_B S_B U_B^\top$ and similarly for $C$. Employing this identity, we see that

\begin{align}
\begin{split}
((\alpha \beta)^{-1} I_{PQ} & + U\T U \otimes V\T V) \\
& =  \left(U_V \otimes U_U\right)
  \invb{\invb{\alpha \beta} I_{PQ} + S_V \otimes S_U}
  \left(U_V\T \otimes U_U\T\right)
\end{split}
\end{align}

where the eigen-decomposition of $V\T V$ is $U_V S_V U_V\T$ and similarly for $U\T U$. For brevity in the following we denote $S_{VU} = \left(\invb{\alpha \beta} I_{PQ} + S_V \otimes S_U\right)$. Note that being a diagonal matrix, its inverse is trivially obtained. This allows us to re-write the solution for $M_Y$ as

\begin{align}
\vecf{M_Y} 
    & = \invb{\invb{\alpha \beta} I_{PQ} + U\T U \otimes V\T V}\vecf{U\T M_A V} \\
    & = \left(U_V \otimes U_U\right) \inv{S_{VU}} \left(U_V \otimes U_U\right)\T \vecf{U\T M_A V}
\end{align}

By first using the fact that $\inv{S_{VU}}$ is a diagonal matrix, and then the fact that $(C\T \otimes A)\vecf{B} = \vecf{ABC}$ we can simplify the expression 
\begin{align}
 &\inv{S_{VU}}(U_V^\top \otimes U_U^\top) \vecf{U\T M_A V} \\
= & (U_V^\top \otimes U_U^\top)\inv{S_{VU}} \vecf{U\T M_A V} \\
= & (U_V^\top \otimes U_U^\top)\vecf{\diag{\inv{S_{VU}}}^{(P)} .* (U\T M_A V)} \\
= & \vecf{U_U (\diag{\inv{S_{VU}}}^{(P)} .* (U\T M_A V)) U_V^\top} = \vecf{\hat S}
\end{align}

where $.*$ indicates the Hadamard product and the notatiuon $A^{(N)}$ indicates the N-th order \emph{vec-transpose}\cite{Minka2000a} of the matrix $A$, which for a vector is equivalent to a fortran-order reshaping of its contents to form a matrix containing contain $N$ rows.

Once again applying the identity $(C\T \otimes A)\vecf{B} = \vecf{ABC}$ we now can re-write the full update for $\vecf{M_Y}$ as:

\begin{align}
\vecf{M_Y} & = (U_V \otimes U_U)\vecf{\hat S} = \vecf{U_U^\top \hat S U_V}
\end{align}

And then remove the $\vecf{\cdot}$ function from both sides and thus obtain our final solution:

\begin{align}
M_Y = U_U^\top U_U (\diag{\inv{S_{VU}}}^{(P)} .* (U\T M_A V)) U_V^\top U_V
\end{align}

where `.*` indicates the Hadamard product, and $\hat S$ is an appropriately shaped matrix containing the elements of the diagonal given by $(\alpha I + S_V \otimes S_U)$. The time complexity of this approach is $O(P^3 + Q^3)$ rather than the initial approach in $\eqref{eqn:dumb-y-soln}$ which is $O(P^3Q^3)$

\fixme{What this does mean. }

\subsection{q(A)}
Denoting $M_\Theta = \{ \vv{m}_d \}_{d=1}^D$, and taking derivatives we arrive at the solution

\begin{align}
- \inv\tau\inv\sigma M_A 
- \inv\sigma X\T X M_A
+ \inv\tau \inv\sigma UM_YV\T
+ \inv\sigma X\T M_\Theta
\mbeq 0\end{align}\begin{align}
\implies M_A = \invb{\inv\tau I_F + X\T X}\left(\inv\tau UM_YV\T + X\T M_\Theta \right)
\end{align}

Similarly taking derivatives gives the following updates for column and row covariances $S_A$ and $R_A$. 
\begin{align}
\inv{S_A}
    & = \oneover{K} \left( \Tr{\inv\sigma I_K R_A}\inv\tau I_F + \Tr{\inv\sigma I_K R_A} X\T X  \right) \\
    & = \oneover{K}  \Tr{(\inv\sigma I_K) R_A}\left(\inv\tau I_F + X\T X\right) \\
\inv{R_A} 
    & = \oneover{F} \left( \Tr{\inv\tau I_F S_A}\inv\sigma I_K + \Tr{X\T X S_A}\inv\sigma I_K \right)\\
    & = \oneover{F} \Tr{\left(\inv\tau I_F + X\T X \right) S_A}\inv\sigma I_K
\end{align}
At first glance, these appear coupled, however if we substitute the solution for $R_A$ into $S_A$ we obtain

\begin{align}
\inv{S_A} & = \oneover{K} \Tr{
    \frac{
        \inv\sigma 
     } {
         \inv\sigma \Tr{\left(\inv\tau I_F + X\T X \right) S_A}
     }
     I_K
} \left(\inv\tau I_F + X\T X \right) \\
 &= \oneover{
         \Tr{\left(\inv\tau I_F + X\T X \right) S_A}
     } \left(\inv\tau I_F + X\T X \right)
\end{align}

Pre-multiplying the equation on the left by $\Tr{\left(\inv\tau I_F + X\T X \right) S_A} S_A$ we get
\begin{align}
 \Tr{\left(\inv\tau I_F + X\T X \right) S_A} I_F = S_A \left(\inv\tau I_F + X\T X \right)
\end{align}
which implies that $S_A \left(\inv\tau I_F + X\T X \right)$ is a multiple of the identity matrix, 
specifically there exists a family of solutions $\inv{S_A} = \left\{ q\left(\inv\tau I_F + X\T X \right) | q \in \mathbb{R}, q \neq 0 \right\}$. 
Choosing $q = 1$ and proceeding similarly for $R_A$ we arrive at the simpler pair of updates:

\begin{align}
S_A & = \invb{\inv\tau I_F + X\T X} &
R_A & = \sigma I_K
\end{align}

As the posterior row-covariance is the same as the prior, it cancels out in the update for $M_A$. \fixme{Write up the kronecker learning issue}

\subsection{q($\Theta$)}
Having applied the bound, taking derivatives and setting to zero gives the following solution for the posterior mean $\md$

\begin{align}
- \inv\sigma I_K \md - n_d H \md + n_d \vv{b}_d + \inv\sigma M_A\T\xd + \vv{\gamma}_{d\cdot} \mbeq 0
\end{align}
\begin{align}
\implies \md = \invb{\inv\sigma I_K + n_d H}\left(\inv\sigma M_A\T\xd + n_d \vv{b}_d + \vv{\gamma}_{d\cdot} \right)
\end{align}

Similarly the posterior covariance, which is assumed to be diagonal, is

\begin{align}
s_{dk}^{-1} \quad =  \quad \inv\sigma + n_d H_{kk} \quad = \quad \inv\sigma + n_d \left(\frac{K-1}{2K}\right)
\end{align}

\subsection{q(Z)}
Adding a language multiplier to enforce the constraint that $\sum_k \gdnk = 1$ and collecting terms leads to the solution:

\begin{align}
& \frac{d}{d \gdnk} \gdnk m_{dk} + \gdnk \ln \phi_{k,w_{dn}} - \gdnk \ln \gdnk + \lambda (1 - \sum_j \gamma_{dnj}) \\
& = m_{dk} + \ln \phi_{k,w_{dn}} -\lambda - \ln \gdnk - 1 \mbeq 0
\end{align}
\begin{align}
\implies \gdnk = \oneover{e^\lambda} \exp\left(m_{dk} + \ln \phi_{k,{w_{dn}}} -1 \right)
\end{align}

Using the constraint that $\sum_k \gdnk = 1$ we solve for $\lambda$ and plug in the solution to obtain
\begin{align}
\gdnk = \frac{e^{m_{dk}} \phi_{k,w_{dn}}}{\sum_j e^{m_{dj}} \phi_{j,w_{dn}}}
\end{align}

\section{Parameters}
\subsection{U and V}
The updates for U and V are straightforwardly obtained by taking derivatives and setting to zero:

\begin{align}
U & = M_A V M_Y\T \invb{M_Y V\T V MY\T} &
V & = M_A\T U M_Y \invb{M_Y\T U\T U M_Y} \label{eqn:app-v-update}
\end{align}

There is an identifiability concern with these parameters however, as for any $\alpha$, if we scale $U' = \alpha U$ and $V' = \oneover{\alpha} V$, then $U'Y{V'}\T = U Y V\T$. To address this put a tight Gaussian priors on $\nor{\vv{0}}{0.00001 I}$ on both U and V to render them weakly identifiable\fixme{ref here}.

To rectify this, we enforce a constraint that the top-left element of $U$ should always be 1, essentially setting $\alpha = \oneover{U_{11}}$. In practice this leads to the following update cycle for $U$.

\begin{align}
U^* &= M_A V M_Y\T \invb{M_Y V\T V MY\T} &
U &= \frac{1}{U^*_{11}} U^*
\end{align}

with V updated as in \eqref{eqn:app-v-update}. This is similar to the \fixme{name} method used to \fixme{how} identify the covariance matrices for a matrix-variate normal distribution.


\subsection{$\phi_k$}
The derivation of the update for $\pk$ proceeds similarly to $\gdnk$, with the use of a Lagrange multiplier to ensure the constraint that $\sum_t \phi_{kt} = 1$. This gives the initial solution
\begin{align}
\phi_{kt} = \oneover{\lambda} \sum_d \sum_n \sum_k \gdnk \wdnt
\end{align}
From which the the final answer is obtained
\begin{align}
\phi_{kt} = \frac{\sum_d \sum_n \sum_k \gdnk \wdnt}{\sum_v \sum_d \sum_n \sum_k \zdnk w_{dnv}}
\end{align}

In practice, to avoid word probabilities collapsing to zero, we put a Dirichlet prior on $\pk$ with a symmetric hyperparameter $\vv{\beta}$, and use maximum-a-posteriori estimation to get the update
\begin{align}
\phi_{kt} = \frac{(\beta_t - 1) + \sum_d \sum_n \sum_k \gdnk \wdnt}{\sum_v (\beta_v - 1) + \sum_d \sum_n \sum_k \gdnk w_{dnv}} \label{eqn:app-phi-kt}
\end{align}

We avoid full variational inference for performance reasons, in particular the fact that \eqref{eqn:app-phi-kt} is trivially vectorised.

\subsection{$\xi_d$}
The terms involving the bound parameter (once $\vv{b}_d$ and $c_d$ are expanded) are

\begin{align}
n_d \left( 
    \ed\T H \md
    - \vv{\sigma}\left(\ed\right)\T \ md 
        - \half\ed\T H \ed 
        + \vv{\sigma}\left(\ed\right)\T \ed 
        - \lse\left(\ed\right)
\right)
\end{align}

Making use of the following identites:
\begin{align}
\nabla_{\vv{\theta}}\vv{\sigma}\left(\vv{\theta}\right) & = \diag{\vv{\sigma}\left(\vv{\theta}\right) } \left(\one - \vv{\sigma}\left(\vv{\theta}\right) \right) &
\nabla_{\vv{\theta}}\lse\left(\vv{\theta}\right) & = \vv{\sigma}\left(\vv{\theta}\right) 
\end{align}

We can derivatives and set to zero to obtain the following equation:

\fixme{Left Align}
\begin{align}
\begin{split}
& H\md
- \md\T \diag{\vv{\sigma}\left(\vv{\ed}\right) } \left(\one - \vv{\sigma}\left(\vv{\ed}\right) \right)
- H \ed 
+ \vv{\sigma}\left(\vv{\ed}\right) \\
& + \vv{\sigma}\left(\ed\right)\T \diag{\vv{\sigma}\left(\vv{\ed}\right) } \left(\one - \vv{\sigma}\left(\vv{\ed}\right) \right)
- \vv{\sigma}\left(\vv{\ed}\right) 
\qquad\qquad\quad\mbeq 0
\end{split}
\end{align}

Simplifying we obtain
\begin{align}
H(\md -  \ed) = (\md + \vv{\sigma}(\ed))\nabla_\ed \vv{\sigma}(\ed)
\end{align}




\bibliographystyle{plain}
\bibliography{/Users/bryanfeeney/Documents/library.bib}

\end{document}


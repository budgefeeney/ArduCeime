\documentclass[10pt,fleqn]{article}

\usepackage[english]{babel}
\usepackage[utf8x]{inputenc}
\usepackage{enumerate}
\usepackage{amsmath}
\usepackage{amssymb}
\usepackage{amsfonts} 
\usepackage{mathtools}
\usepackage{graphicx}
\usepackage{bm}
\usepackage[usenames,dvipsnames]{color}
\usepackage{todonotes}
\usepackage{dsfont}
\usepackage{hyperref}
\hypersetup{
    colorlinks,
    citecolor=black,
    filecolor=black,
    linkcolor=black,
    urlcolor=black
}
\usepackage{algorithm}
\usepackage{algorithmic}
\usepackage{appendix}
\usepackage{subcaption}
\usepackage{fancyvrb}
\usepackage{subfigure}
\usepackage{graphicx,xcolor}
\usepackage{pifont,mdframed}
\usepackage{tikz}
\usepackage{bm}
\usetikzlibrary{fit,positioning}


%
% Macros
%
\newcommand \cashort[1] { {\todo[color=yello]{#1 -- Cedric}} }
\newcommand \calong[1]  { { \todo[inline,color=yellow]{#1 -- Cedric} } }
\newcommand \gbshort[1] { {\todo[color=cyan!40]{#1 -- Guillaume}} }
\newcommand \gblong[1]  { { \todo[inline, color=cyan!40]{#1 -- Guillaume} } }
\newcommand \mgshort[1] { {\todo{#1 -- Mark}} }
\newcommand \mglong[1]  { { \todo[inline]{#1 -- Mark} } }
\newcommand \bfshort[1] { {\todo[color=green!40]{#1 -- Bryan}} }
\newcommand \bflong[1]  { { \todo[inline,color=green!40]{#1 -- Bryan} } }


% Adds a plus const to the end of a math expression
\def \pcst{+\text{const}}

% A fancy version for capital R
\def \Rcal{\mathcal{R}}

% A fancy version for r
\def \rcal{\mathbf{r}}

% Loss function / log likelihood as appropriate
\def \L{\mathcal{L}}

% KL divergence [Math Mode]
\newcommand{\kl}[2] {
	\text{KL}\left[#1||#2\right]
}

\newcommand \vecf[1] {
    \text{vec}\left(#1\right)
}

\newcommand \ent[1] {
    \text{H} \left[ #1 \right]
}

\newcommand \mut[2] {
    \text{I} \left[ #1 ; #2 \right]
}

\newcommand \dvi[2] {
    \text{D}_\text{VI} \left[ #1; #2 \right]
}

% Starts an expected value expresses [Math Mode]
\newcommand{\starte}[1] {%
	\mathbb{E}_{#1}\left[
}

% Ends an expected value expression [Math Mode]
\def \ende{\right]}

% Starts an varianc expresses [Math Mode]
\newcommand{\startv}[1] {%
	\mathbb{V}\text{ar}_{#1}\left[
}

% Ends an variance expression [Math Mode]
\def \endv{\right]}

%\newcommand \ex[2] {
%    \bigl\langle #1 \bigr\rangle_{#2}
%}
\newcommand \ex[2] {
    \mathbb{E}_{ { #2 } }\left[ #1 \right]
}
\newcommand \var[2] {
    \mathbb{V}ar_{ { #2 } }\left[ #1 \right]
}

\newcommand \halve[1] {
	\frac{#1}{2}
}

\newcommand \half {
    \halve{1}
}

\newcommand \tr { \text{tr} } 

\newcommand \T { ^\top } 

\newcommand \fixme[1] {
    {\color{red} FIXME: #1}
}

\newcommand \vv[1] { \bm #1 }

\newcommand{\mbeq}{\overset{!}{=}}

\newcommand \diag[1] { \text{diag} \left( {#1} \right) }
\newcommand \diagonal[1] { \text{diagonal} \left( {#1} \right) }

\newcommand \Ed {{ \vv{\xi}_d}}
\newcommand \Edj {{\xi_{dj}}}
\newcommand \Edk {{\xi_{dk}}}
\newcommand \AEdj {{\Lambda(\xi_{dj})}}
\newcommand \AEdk {{\Lambda(\xi_{dk})}}
\newcommand \AEd  {{ \bm{\Lambda}(\bm{\xi}_d) }}

\newcommand \Axi { { \Lambda_{\xi} } }
\newcommand \bxi { { \vv{b}_{\xi} } }
\newcommand \cxi { { c_{\xi} } }


\newcommand \wdoc      { { \vv{w}_d } }
\newcommand \wdt[0]  { { w_{dt} } }
\newcommand \wdn[0]  { { \vv{w}_{dn} } }
\newcommand \wdnt[0]  { { w_{dnt} } }
\newcommand \wdd[0]   { { \vv w_{d} } }
\newcommand \zd[0]   { { \vv z_{d} } }
\newcommand \zdn[0]  { { \vv{z}_{dn} } }
\newcommand \zdnk[0] { { z_{dnk} } }
\newcommand \zdk[0]  { { z_{dk} } }
\newcommand \thd[0]  { { \vv \theta_d } }
\newcommand \thdk[0] { { \theta_{dk} } }
\newcommand \thdj[0] { { \theta_{dj} } }
\newcommand \epow[1] { { e^{#1} } }
\newcommand \pkt     { { \phi_{kt}  } }
\newcommand \pk      { { \vv \phi_k } }
\newcommand \lmd     { { \vv \lambda_d } }
\newcommand \lmdk    { { \lambda_{dk} } }
\newcommand \xd      { { \vv x_d } }
\newcommand \atxd     { A ^\top \bm x_d}
\newcommand \axd     { A\bm x_d}
\newcommand \tsq      { { \tau^2 } }
\newcommand \ssq      { { \sigma^2 } }
\newcommand \tmsq     { { \tau^{-2} } }
\newcommand \asq      { { \alpha^2 } }
\newcommand \amsq     { { \alpha^{-2} } }
\newcommand \sgsq     { { \sigma^2 } }
\newcommand \xvec     { { \vv{x} } }
\newcommand \omk      { { \bm \omega _k } }
\newcommand \omkt     { { \omega_{kt} } }
\newcommand \oma     { { \Omega_A } }
\newcommand \gdn      { { \vv{\gamma}_{dn} } }
\newcommand \gdnk     { { \gamma_{dnk} } }
\newcommand \gdk      { { \gamma_{dk} } }
\newcommand \isigt   { { \Sigma^{-1}_{\bm \theta} } }




\newcommand \halfSig { \frac{1}{2\sigma^2} }

\newcommand \nor[2]   { \mathcal{N} \left( {#1}, {#2} \right) }
\newcommand \nord[3]   { \mathcal{N}_{#1} \left( {#2}, {#3} \right) }
\newcommand \mnor[3]  { \mathcal{N} \left(#1, #2, #3\right) }
\newcommand \norp[3]  { \mathcal{N} \left(#1; #2, #3\right) }
\newcommand \mnorp[4] { \mathcal{N} \left(#1; #2, #3, #4\right) }
\newcommand \mul[1]   { \mathcal{M} \left( {#1} \right) }
\newcommand \muln[2]  { \mathcal{M} \left( {#1},{#2} \right) }
\newcommand \dir[1]   { \mathcal{D} \left( {#1} \right) }
\newcommand \pois[1]  { \mathcal{P} \left( {#1} \right) }
\newcommand \gp[2]    { \mathcal{GP} \left( {#1}, #2 \right) }
\newcommand \dir[1]   { \mathcal{D} \left( {#1} \right) }
\newcommand \gam[2]   { \mathcal{G} \left( {#1}, {#2} \right) }
\newcommand \beta[1]  { \mathcal{B}eta \left( {#1}, {#2} \right) }

\newcommand \lne[1]  { { \ln \left( 1 + e^{ #1 } \right) } }
\newcommand \Tr[1]   { \tr \left(  {#1}  \right) }

\newcommand \roud  { \vv{\rho}_{d}  }
\newcommand \rodk { \rho_{dk} }

\newcommand \exA[1]  { \ex{#1}{q(A)} }
\newcommand \exV[1]  { \ex{#1}{q(V)} }
\newcommand \exT[1]  { \ex{#1}{q(\Theta)} }
\newcommand \extd[1] { \ex{#1}{q(\thd)} }
\newcommand \exTV[1] { \ex{#1}{q(\Theta)q(V)} }

\newcommand \Real[0]  { { \mathbb{R} } }
\newcommand \VReal[1] { { \mathbb{R}^{#1} } }
\newcommand \MReal[2] { { \mathbb{R}^{#1 \times #2} } }
\newcommand \Nat[0]  { { \mathbb{N} } }
\newcommand \VNat[1] { { \mathbb{N}^{#1} } }
\newcommand \MNat[2] { { \mathbb{N}^{#1 \times #2} } }

\newcommand \inv[1] { {#1}^{-1} }
\newcommand \invb[1] { \inv{\left( #1 \right)} }

\newcommand \cn { \textsuperscript{\texttt{[{\color{blue}Citation Needed}]}} }

\newcommand \const { { \text{c} } }

\providecommand \floor [1] { \left \lfloor #1 \right \rfloor }
\providecommand \ceil [1] { \left \lceil #1 \right \rceil }


\newcommand \vt[2] { { #1^{(#2)} } }

\newcommand \hashtag[1] { { \ttfamily \##1 } }

\newcommand \mvy  { \vv{m}_{\vv{y}} }
\newcommand \sigvy { { S_Y } }

\newcommand \mmy  { M_Y      }
\newcommand \md   { \vv{m}_d }
\newcommand \phin { \vv{\phi}_n }
\newcommand \isigma { { \inv{\Sigma} } }

\newcommand \sigv     { { \Sigma_V } }
\newcommand \isigv     { { \Sigma^{-1}_V } }

\newcommand \sigy { { \Sigma_Y } }
\newcommand \isigy { { \Sigma_{-1}_Y } }


\newcommand \omy  { { \Omega_Y } }
\newcommand \iomy { { \inv{\Omega_Y} } }

\newcommand \siga     { { \Sigma_A } }
\newcommand \isiga     { { \Sigma^{-1}_A } }
\newcommand \diagv { { \diag{\nu_1,\ldots,\nu_P} } }

\newcommand \ma { \vv{m}_a }
\newcommand \my { \vv{m}_y }

\newcommand \VoU { V \otimes U }

\newcommand \one { \mathbb{1} }
%\newcommand \one  {{  \mathds{1} }}

\newcommand \lse { \text{lse} }
%\newcommand \lse[0] { \mathrm{lse} }

% Conditional independence 
\def\ci{\perp\!\!\!\perp} % from Wikipedia



% ------ For the eval section

% Multinomial PDF [Math Mode]
% params: 1 - the variable
%         2 - the value
%         3 - the state indicator (e.g. k for a distro with K values)
%         4 - any additional subscript
\newcommand{\mpdf}[4] {
	\prod_{#3} {#1}_{{#4} {#3}} ^ {#2}
}

% Dirichlet PDF [Math Mode]
% params: 1 - the variable
%         2 - the hyper-parameter
%         3 - the state indicator (e.g. k for a distro with K values)
%         4 - any additional subscript
\newcommand{\dpdf}[4] {
	\frac{1}{B({#2})} \prod_{#3} {#1}_{{#4} {#3}} ^ {({#2}_{#3} - 1)}
}

% To simplify the sampling equations, this is indicates
% that the given value has had datapoint "m" stripped out
%
\newcommand{\lm}[1] {
	#1^{\setminus m}
}

\newcommand \model[0] {
    \mathcal{M}
}

\newcommand \perplexity[1] {
    \mathcal{P} \left( { #1 } \right)
}

\newcommand \WTrain {
    \mathcal{W}^{(t)}
}

\newcommand \WQuery {
    \mathcal{W}^{(q)}
}

\newcommand \oneover[1] {
    \frac{1}{ {#1} }
}

\newcommand \samp[1] {
    { #1 }^{(s)}
}

\newcommand \etd[0] {
    \vv{\eta}_d
}

\begin{document}



\section{Dirichlet Processes}

\subsection{Bayesian Historgram}
Consider the issue of estimating a density using a histogram. If one is \emph{given} a partitioning scheme $\xi_0 < \xi_1 < \ldots < \xi_K$ then the probability of a value is given by
\begin{align}
p(y) = \sum_k \one_{\xi_{k-1}<y<\xi_k} \frac{\theta_k}{\xi_k - \xi_{k-1}} \label{chap1:eqn:bayes-hist}
\end{align}
A natural choice for the prior distribution over $\vv{\theta}$ is the Dirichlet parameterised by a vector $\vv{\alpha}$. This parameterisation can be factorised into $\alpha \vv{u}$ where $\vv{u}$ is a vector on the K-1 simplex. Hence
\begin{align}
\vv{\theta} &\sim \dir{\alpha \vv{u}} &
\ex{\theta_k}{} & = u_k = \frac{\alpha u_k}{\sum_j \alpha u_j} &
\var{\theta_k}{} & = \oneover{\alpha + 1} u_k (1 - u_k) \label{chap1:eqn:dir-variance-etc}
\end{align}
From this it can be seen that $\alpha$ acts as a precision, or concentration, parameter. Higher values of $\alpha$ will prefer draws closer to $\vv{u}$.

There are two standard ways of taking a draw from a Dirichlet. One takes $K$ draws $\hat{\theta}_k \sim \gam{\alpha u_k}{\lambda}$ -- where $\lambda$ can be any value -- and then normalises them so that $\theta_k = \frac{\hat{\theta}_k}{\sum_j \hat{\theta}_j}$. 

A different approach is obtained by observing that $\vv{\theta} \sim \dir{\alpha \vv{u}}$ the marginal distribution of a single component is $\theta_k \sim \betadist{\alpha u_k}{1- \alpha u_k}$, one can alternatively sample $\theta$ by repeatedly sampling from Beta distributions
\begin{align}
V_k & \sim \betadist{\alpha u_k}{\alpha \sum_{j>k} u_j} &
\theta_k & = V_k \prod_{j > k}(1 - V_j) &
\theta_K = 1 - \sum_{j=1}^{K} \theta_j
\end{align}
This is known as a stick-breaking construction. We begin with a stick of unit length, then break off a portion given by a draw from a Dirichlet distribution. We then take the remainder and break a scaled portion from that. We continue till we've reached the K-th element, at which point the \emph{entirety} of the remaining stick length is assigned to $\theta_K$.

One might reasonably ask what happens in the limit as $K \rightarrow \infty$. If our stick-breaking construction were extended, we would end up with infinitismally small draws.

\bflong{Infinitely long Dirichlet Distribution}

\newcommand \sampspace { { \Omega } }
\newcommand \borel { { \mathcal{B} } }

\subsection{Dirichlet Process}
Consider the standard definition of a probability space $(\sampspace, \borel, P)$ where $\sampspace $ is the sample space, $\borel$ is a $\sigma$-algebra -- i.e. a collection of sets closed under complement, intersection and countably infinite union -- and $P$ is a probability measure $P : \Sigma \rightarrow [0,1]$. For a finite subset of sets $B_k$ drawn from $\borel$ we can assign probabilities to these sets, or ``bins" according to

\begin{align}
P(B_1), \ldots, P(B_K) & \sim \dir{\alpha P_0 (B_1), \ldots, P_0 (B_K)} \end{align}
This presentation differs to \eqref{chap1:eqn:bayes-hist} in that it does not specify how the probability mass $P(B_k)$ should be distributed across the values in $B_k$. The base measure $P_0$ is also defined in $(\sampspace, \borel)$ and can be any distribution, e.g. a Gaussian in a mixture model. Standard conjugacy results provide that

\begin{align}
\begin{split}
P(B_1), \ldots, &P(B_k) | x_1, \ldots x_n
= \\
& \dir{\alpha P_0(B_1) + \sum_i \one_{y_i \in B_1}P_0(B_1), \ldots, \sum_i \one_{y_i \in B_K} P_0(B_K)}
\end{split}
\end{align}

On its own, this does not define a complete prior on $P$ as it is conditioned on a specific partition $B_1, \ldots, B_K$. Generalising this model\bfshort{A reference would be really handy here!} to handles all possible partitions $B_1, \ldots, B_K$, for all possible values of $K$, gives rise to the Dirichlet process.

\begin{align}
P \sim \DP{\alpha P_0}
\ex{P(B_k)}{} &= P_0 (B_k) &
\var{P(B_k)}{} &= \oneover{1 + \alpha} P_0(B_k) (1 - P_0 (B_k) )
\end{align}
The Dirichlet Process can be expressed using a Beta distribution
\begin{align}
P(B_k) = \betadist{\alpha P_0(B_k)}{1 - \alpha P_0(B_k)}
\end{align}
and is conjugate to a multinomial likelihood:




\bibliographystyle{plain}
\bibliography{/Users/bryanfeeney/Documents/library.bib}

\end{document}

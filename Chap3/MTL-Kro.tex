\documentclass[10pt,fleqn]{article}

\usepackage[english]{babel}
\usepackage[utf8x]{inputenc}
\usepackage{enumerate}
\usepackage{amsmath}
\usepackage{amssymb}
\usepackage{amsfonts} 
\usepackage{mathtools}
\usepackage{graphicx}
\usepackage{bm}
\usepackage[usenames,dvipsnames]{color}
\usepackage{todonotes}
\usepackage{dsfont}
\usepackage{hyperref}
\hypersetup{
    colorlinks,
    citecolor=black,
    filecolor=black,
    linkcolor=black,
    urlcolor=black
}
\usepackage{algorithm}
\usepackage{algorithmic}
\usepackage{appendix}
\usepackage{subcaption}
\usepackage{fancyvrb}
\usepackage{subfigure}
\usepackage{graphicx,xcolor}
\usepackage{pifont,mdframed}
\usepackage{tikz}
\usepackage{bm}
\usetikzlibrary{fit,positioning}


%
% Macros
%
\newcommand \cashort[1] { {\todo[color=yello]{#1 -- Cedric}} }
\newcommand \calong[1]  { { \todo[inline,color=yellow]{#1 -- Cedric} } }
\newcommand \gbshort[1] { {\todo[color=cyan!40]{#1 -- Guillaume}} }
\newcommand \gblong[1]  { { \todo[inline, color=cyan!40]{#1 -- Guillaume} } }
\newcommand \mgshort[1] { {\todo{#1 -- Mark}} }
\newcommand \mglong[1]  { { \todo[inline]{#1 -- Mark} } }
\newcommand \bfshort[1] { {\todo[color=green!40]{#1 -- Bryan}} }
\newcommand \bflong[1]  { { \todo[inline,color=green!40]{#1 -- Bryan} } }


% Adds a plus const to the end of a math expression
\def \pcst{+\text{const}}

% A fancy version for capital R
\def \Rcal{\mathcal{R}}

% A fancy version for r
\def \rcal{\mathbf{r}}

% Loss function / log likelihood as appropriate
\def \L{\mathcal{L}}

% KL divergence [Math Mode]
\newcommand{\kl}[2] {
	\text{KL}\left[#1||#2\right]
}

\newcommand \vecf[1] {
    \text{vec}\left(#1\right)
}

\newcommand \ent[1] {
    \text{H} \left[ #1 \right]
}

\newcommand \mut[2] {
    \text{I} \left[ #1 ; #2 \right]
}

\newcommand \dvi[2] {
    \text{D}_\text{VI} \left[ #1; #2 \right]
}

% Starts an expected value expresses [Math Mode]
\newcommand{\starte}[1] {%
	\mathbb{E}_{#1}\left[
}

% Ends an expected value expression [Math Mode]
\def \ende{\right]}

% Starts an varianc expresses [Math Mode]
\newcommand{\startv}[1] {%
	\mathbb{V}\text{ar}_{#1}\left[
}

% Ends an variance expression [Math Mode]
\def \endv{\right]}

%\newcommand \ex[2] {
%    \bigl\langle #1 \bigr\rangle_{#2}
%}
\newcommand \ex[2] {
    \mathbb{E}_{ { #2 } }\left[ #1 \right]
}
\newcommand \var[2] {
    \mathbb{V}ar_{ { #2 } }\left[ #1 \right]
}

\newcommand \halve[1] {
	\frac{#1}{2}
}

\newcommand \half {
    \halve{1}
}

\newcommand \tr { \text{tr} } 

\newcommand \T { ^\top } 

\newcommand \fixme[1] {
    {\color{red} FIXME: #1}
}

\newcommand \vv[1] { \bm #1 }

\newcommand{\mbeq}{\overset{!}{=}}

\newcommand \diag[1] { \text{diag} \left( {#1} \right) }
\newcommand \diagonal[1] { \text{diagonal} \left( {#1} \right) }

\newcommand \Ed {{ \vv{\xi}_d}}
\newcommand \Edj {{\xi_{dj}}}
\newcommand \Edk {{\xi_{dk}}}
\newcommand \AEdj {{\Lambda(\xi_{dj})}}
\newcommand \AEdk {{\Lambda(\xi_{dk})}}
\newcommand \AEd  {{ \bm{\Lambda}(\bm{\xi}_d) }}

\newcommand \Axi { { \Lambda_{\xi} } }
\newcommand \bxi { { \vv{b}_{\xi} } }
\newcommand \cxi { { c_{\xi} } }


\newcommand \wdoc      { { \vv{w}_d } }
\newcommand \wdt[0]  { { w_{dt} } }
\newcommand \wdn[0]  { { \vv{w}_{dn} } }
\newcommand \wdnt[0]  { { w_{dnt} } }
\newcommand \wdd[0]   { { \vv w_{d} } }
\newcommand \zd[0]   { { \vv z_{d} } }
\newcommand \zdn[0]  { { \vv{z}_{dn} } }
\newcommand \zdnk[0] { { z_{dnk} } }
\newcommand \zdk[0]  { { z_{dk} } }
\newcommand \thd[0]  { { \vv \theta_d } }
\newcommand \thdk[0] { { \theta_{dk} } }
\newcommand \thdj[0] { { \theta_{dj} } }
\newcommand \epow[1] { { e^{#1} } }
\newcommand \pkt     { { \phi_{kt}  } }
\newcommand \pk      { { \vv \phi_k } }
\newcommand \lmd     { { \vv \lambda_d } }
\newcommand \lmdk    { { \lambda_{dk} } }
\newcommand \xd      { { \vv x_d } }
\newcommand \atxd     { A ^\top \bm x_d}
\newcommand \axd     { A\bm x_d}
\newcommand \tsq      { { \tau^2 } }
\newcommand \ssq      { { \sigma^2 } }
\newcommand \tmsq     { { \tau^{-2} } }
\newcommand \asq      { { \alpha^2 } }
\newcommand \amsq     { { \alpha^{-2} } }
\newcommand \sgsq     { { \sigma^2 } }
\newcommand \xvec     { { \vv{x} } }
\newcommand \omk      { { \bm \omega _k } }
\newcommand \omkt     { { \omega_{kt} } }
\newcommand \oma     { { \Omega_A } }
\newcommand \gdn      { { \vv{\gamma}_{dn} } }
\newcommand \gdnk     { { \gamma_{dnk} } }
\newcommand \gdk      { { \gamma_{dk} } }
\newcommand \isigt   { { \Sigma^{-1}_{\bm \theta} } }




\newcommand \halfSig { \frac{1}{2\sigma^2} }

\newcommand \nor[2]   { \mathcal{N} \left( {#1}, {#2} \right) }
\newcommand \nord[3]   { \mathcal{N}_{#1} \left( {#2}, {#3} \right) }
\newcommand \mnor[3]  { \mathcal{N} \left(#1, #2, #3\right) }
\newcommand \norp[3]  { \mathcal{N} \left(#1; #2, #3\right) }
\newcommand \mnorp[4] { \mathcal{N} \left(#1; #2, #3, #4\right) }
\newcommand \mul[1]   { \mathcal{M} \left( {#1} \right) }
\newcommand \muln[2]  { \mathcal{M} \left( {#1},{#2} \right) }
\newcommand \dir[1]   { \mathcal{D} \left( {#1} \right) }
\newcommand \pois[1]  { \mathcal{P} \left( {#1} \right) }
\newcommand \gp[2]    { \mathcal{GP} \left( {#1}, #2 \right) }
\newcommand \dir[1]   { \mathcal{D} \left( {#1} \right) }
\newcommand \gam[2]   { \mathcal{G} \left( {#1}, {#2} \right) }
\newcommand \beta[1]  { \mathcal{B}eta \left( {#1}, {#2} \right) }

\newcommand \lne[1]  { { \ln \left( 1 + e^{ #1 } \right) } }
\newcommand \Tr[1]   { \tr \left(  {#1}  \right) }

\newcommand \roud  { \vv{\rho}_{d}  }
\newcommand \rodk { \rho_{dk} }

\newcommand \exA[1]  { \ex{#1}{q(A)} }
\newcommand \exV[1]  { \ex{#1}{q(V)} }
\newcommand \exT[1]  { \ex{#1}{q(\Theta)} }
\newcommand \extd[1] { \ex{#1}{q(\thd)} }
\newcommand \exTV[1] { \ex{#1}{q(\Theta)q(V)} }

\newcommand \Real[0]  { { \mathbb{R} } }
\newcommand \VReal[1] { { \mathbb{R}^{#1} } }
\newcommand \MReal[2] { { \mathbb{R}^{#1 \times #2} } }
\newcommand \Nat[0]  { { \mathbb{N} } }
\newcommand \VNat[1] { { \mathbb{N}^{#1} } }
\newcommand \MNat[2] { { \mathbb{N}^{#1 \times #2} } }

\newcommand \inv[1] { {#1}^{-1} }
\newcommand \invb[1] { \inv{\left( #1 \right)} }

\newcommand \cn { \textsuperscript{\texttt{[{\color{blue}Citation Needed}]}} }

\newcommand \const { { \text{c} } }

\providecommand \floor [1] { \left \lfloor #1 \right \rfloor }
\providecommand \ceil [1] { \left \lceil #1 \right \rceil }


\newcommand \vt[2] { { #1^{(#2)} } }

\newcommand \hashtag[1] { { \ttfamily \##1 } }

\newcommand \mvy  { \vv{m}_{\vv{y}} }
\newcommand \sigvy { { S_Y } }

\newcommand \mmy  { M_Y      }
\newcommand \md   { \vv{m}_d }
\newcommand \phin { \vv{\phi}_n }
\newcommand \isigma { { \inv{\Sigma} } }

\newcommand \sigv     { { \Sigma_V } }
\newcommand \isigv     { { \Sigma^{-1}_V } }

\newcommand \sigy { { \Sigma_Y } }
\newcommand \isigy { { \Sigma_{-1}_Y } }


\newcommand \omy  { { \Omega_Y } }
\newcommand \iomy { { \inv{\Omega_Y} } }

\newcommand \siga     { { \Sigma_A } }
\newcommand \isiga     { { \Sigma^{-1}_A } }
\newcommand \diagv { { \diag{\nu_1,\ldots,\nu_P} } }

\newcommand \ma { \vv{m}_a }
\newcommand \my { \vv{m}_y }

\newcommand \VoU { V \otimes U }

\newcommand \one { \mathbb{1} }
%\newcommand \one  {{  \mathds{1} }}

\newcommand \lse { \text{lse} }
%\newcommand \lse[0] { \mathrm{lse} }

% Conditional independence 
\def\ci{\perp\!\!\!\perp} % from Wikipedia



% ------ For the eval section

% Multinomial PDF [Math Mode]
% params: 1 - the variable
%         2 - the value
%         3 - the state indicator (e.g. k for a distro with K values)
%         4 - any additional subscript
\newcommand{\mpdf}[4] {
	\prod_{#3} {#1}_{{#4} {#3}} ^ {#2}
}

% Dirichlet PDF [Math Mode]
% params: 1 - the variable
%         2 - the hyper-parameter
%         3 - the state indicator (e.g. k for a distro with K values)
%         4 - any additional subscript
\newcommand{\dpdf}[4] {
	\frac{1}{B({#2})} \prod_{#3} {#1}_{{#4} {#3}} ^ {({#2}_{#3} - 1)}
}

% To simplify the sampling equations, this is indicates
% that the given value has had datapoint "m" stripped out
%
\newcommand{\lm}[1] {
	#1^{\setminus m}
}

\newcommand \model[0] {
    \mathcal{M}
}

\newcommand \perplexity[1] {
    \mathcal{P} \left( { #1 } \right)
}

\newcommand \WTrain {
    \mathcal{W}^{(t)}
}

\newcommand \WQuery {
    \mathcal{W}^{(q)}
}

\newcommand \oneover[1] {
    \frac{1}{ {#1} }
}

\newcommand \samp[1] {
    { #1 }^{(s)}
}

\newcommand \etd[0] {
    \vv{\eta}_d
}

\begin{document}



\subsubsection{Priors with Kronecker-Product Covariance Structure}
An alternative to employing a single multivariate Gaussian prior over many weight \emph{vectors} is to employ a single matrix-variate Gaussian\cite{Gupta1999} prior over a weight \emph{matrix} $W = \left\{\vv{w}_l\right\}_{l=1}^L$ where $W \in \MReal{L}{F}$. This is denoted as as $\mnor{M}{\Omega}{\Sigma}$ where $\Omega \in \MReal{F}{F}$ and $\Sigma \in \MReal{L}{L}$ are called the row and column covariances, $M \in \MReal{L}{F}$ is the mean matrix, and the density is defined as:
\begin{equation}
p(W) = (2\pi)^{KF}|\Sigma|^{-F}|\Omega|^{-K} \Etr{\half \inv{\Sigma}\left(W - M\right) \inv{\Omega} \left(W - M \right)\T}
\end{equation}

where $\Etr{X} = \exp (\Tr{X})$.

This is related to the multivariate normal distribution by the identity $\mnor{M}{\Omega}{\Sigma} = \nor{\vecf{M}}{\Sigma \otimes \Omega}$ where $\vecf{\cdot}$ is the function that converts a matrix to a vector by stacking its columns. 

The row and column covariances $\Omega$ and $\Sigma$ are unidentifiable as $A \otimes B = \lambda A \otimes \frac{1}{\lambda}B$ for any scalar $\lambda$. In practice this is resolved by forcing the one of the elements on the diagonals of the matrices, such as the the last, to 1 which gives rise to the iterative ``flip-flop" algorithm\cite{Srivastava2009} for maximum-likelihood estimation Given N observations of a matrix $X_n \in \MReal{L}{F}$, one can fit a matrix-variate Gaussian of the form $\mnor{M}{\alpha^2 \Omega}{\Sigma}$ using the following update

\begin{align}
\Omega & = \frac{1}{\hat{\Omega}}_{ll} \hat{\Omega}, \qquad\qquad\hat{\Omega} = \frac{1}{N F} \sum_n (\vv{x} - \vv{m}) \inv{\Sigma} (\vv{x} - \vv{m})\T \\
\Sigma & = \frac{1}{\hat{\Sigma}}_{ff} \hat{\Sigma}, \qquad\qquad\hat{\Sigma} = \frac{1}{N L} \sum_n (\vv{x} - \vv{m})\T \inv{\Omega} (\vv{x} - \vv{m}) \\
\alpha^2 & = \frac{1}{NFL} \sum_n (\vv{x} - \vv{m})\T\invb{\Sigma \otimes \Omega}(\vv{x} - \vv{m})
\end{align}
where $\vv{x} = \vecf{X}$, $\vv{m} = \vecf{M}$ and $M=\frac{1}{N} \sum_n X_n$. These updates are iteratively evaluated till convergence.

%Note that $\Omega \in \MReal{L}{L}$ but its estimator is scaled by $\frac{1}{N F}$ and similarly for $\Sigma$. 
It has been proven that if $N > \max(L, F)$ then this procedure should converge on unique estimates of $\Sigma$ and $\Omega$\cite{Srivastava2009}. Similarly, this approach can be extended\cite{Srivastava2009a} to the case of a matrix-variate distribution $\mnor{XZY}{\Omega}{\Sigma}$ where $X$ and $Y$ are known design matrices and $Z$ is a latent matrix, providing unique estimators of $Z$, $\Omega$ and $\Sigma$ under the identifiability condition that the last element of the diagonal in $\Omega$ be set to one.

%Further extensions have been derived separable covariances with Toeplitz\cite{Wirfalt2010} and per-symmetric\cite{Jansson} structures. 

The same algorithm can also be extended to more than two dimensions\cite{Ohlson2011}\cite{Ohlson2013}, creating a multilinear normal distribution over tensors by means of a multi-linear $\vecf{\cdot}$ function with particular emphasis on the ``doubly-separable" case where the covariance is factorized into three covariance matrices.

In the Bayesian case, the use of shrinkage priors, such as the Inverse-Wishart distribution, on the covariances penalizes explosions and implosions\cite{Archambeau2011} and so makes the covariances weakly identifiable.

Matrix-variate priors can be employed to derive efficient algorithms for multi-task regression. Consider a Gaussian process, typically\cite{Bishop2006} presented as:
\begin{align}
\vv{w} \sim & \nor{0}{\alpha^2 I_F} & \vv{t}|\vv{w},X & \sim \nor{X\vv{w}}{\sigma^2 I_N} \\
& & \vv{t} & \sim \nor{\vv{0}}{\sigma^2I_N + K}
\end{align}
where $K = \frac{1}{\alpha^2} X X\T$ and via the kernel trick\cite{Jst2004} can be replaced with any other valid kernel. To extend this to the multi-task case, the single weight vector can be replaced with a weight matrix $W$ and similarly the target vector $t$ is with a matrix $T \in \MReal{L}{F}$. Using the $\vecf{\cdot}$ function to represent these with multivariate distributions and employing standard Gaussian\cite{Bishop2006} and Kronecker\cite{Minka2000a} identities once can derive the following multi-task GP regression algorithm\cite{Bonilla2008}:

\begin{align}
\vecf{W} \sim & \nor{0}{\alpha^2 I_F \otimes I_L} & 
\vecf{T}|W,X & \sim \nor{\vecf{XW}}{\sigma^2 I_N \otimes I_L} \\
& & \vecf{T} & \sim \nor{\vv{0}}{\Sigma \otimes K + I_L \otimes \sigma^2I_N}
\end{align}
Despite employing a matrix-variate prior over functions, the marginal distribution over T is not equivalent to any matrix-variate distribution as the sum in the covariance precludes any factorisation into a Krnoecker product of two matrices. One could recover a matrix-variate distribution in the no-noise case (i.e. $\sigma^2 = 0$) but this leads to additional complications, discussed later.

The mean prediction $\vv{t}_l^*$ for the l-th task given features $\vv{x}^*$ follows the standard form for a GP.

\begin{align}
m(\vv{x}^*)_l & = \left(\vv{k}_l \otimes \vv{k}_x^* \right)\T\inv{C} \vv{t} & C & = \Sigma \otimes K + \sigma^2 I_N \otimes I_L
\end{align}

where $\vv{k}_l$ is the l-th column of the task covariance matrix $\Sigma$ and $\vv{k}_*$ is the usual vector of covariances between the input $\vv{x}_*$ and all other points.

This model has been independently derived in the geostatistics literature, where it is is considered a co-kriging method known as the intrinsic correlation model\cite{Wackernagel1998}. A counter-intuitive property of this model is that in the no-noise case -- where $\sigma^2 = 0$ and so the covariance of $\vecf{T}$ simplifies to $\Sigma \otimes K$) -- no transfer of learning occurs. 

The reason is that in the case maximum of likelihood estimation this leads to an update of $K$ proportional to $-L \ln |K| - N \ln |T\T \inv{K} T|$, an expression that does not involve $\Sigma$, whereas the update for $\Sigma$ is $\frac{1}{N}T\T \inv{K} T$. In short, the targets are correlated using $\Sigma$ and $K$, but we can decorrelate those targets so that only $\Sigma$ remains. This is clearer if we work through the equation for the mean of the predicted target $\vv{t}_* \in \VReal{L}$ for a new input $\vv{x}^*$ in the no noise case:

\begin{align}
m(\vv{x}_*) & = (\Sigma \otimes \vv{k}_*)\T\invb{\Sigma \otimes K}\vecf{T} \\
& = (\Sigma\T \otimes \vv{k}_*\T)\left(\inv{\Sigma} \otimes \inv{K}\right)\vecf{T}\\
& = \left(\Sigma \inv{\Sigma} \right) \otimes \left(\vv{k}_*\T \inv{K}\right)\vecf{T} \\
& = \left( \begin{array}{c}
     \vv{k}_*\T \inv{K} \vv{t}_{\cdot 1} \\
     \vdots \\
     \vv{k}_*\T \inv{K} \vv{t}_{\cdot L} \\
 \end{array}\right) \\
\end{align}

With regard to inference, in published work\cite{Bonilla2008} no flip-flop algorithm has been employed: presumably the use of a parameterised covariance function instead of a covariance matrix ensured that no identifiability issues arose in practice.

%With regard to inference, the authors did not use any of the tricks outlined earlier to handle identifiability: presumably the use of a parameterised covariance function instead of a covariance matrix ensured that no identifiability issues arose in practice, however they did factor the task covariance $\Sigma = L L\T$ according to the Cholesky decomposition to ensure that the resulting estimate of the covariance was always positive semi-definite. 

%Additionally, a number of approximations were used to make calculation of the kernel feasible, such as subselecting a sparse subset of inputs to calculate the kernel matrix\fixme{Nystrom}.

An alternative approach the method outlined above is to introduce a noise-free auxiliary variable, Y, such that $T|Y\sim\nor{\vecf{Y}}{\sigma^2 I_{LF}}$, $Y\sim\nor{\vv{0}}{\Sigma \otimes \Omega}$\cite{Stegle2011}. The advantage of this model is that $Y$ and $T$ therefore have a matrix-variate distributions, and so there's no need to resort to multi-variate forms as outlined above. 

Such a model could further be extended (\cite{Stegle2011} again) to encode a preference for sparse feature by following the Glasso approach\cite{Friedman2008} and putting a Laplace prior on $\vecf{\inv{\Sigma}}$, while representing $\Omega$ as a kernel which captures task correlations. The case of applying the Glasso approach to both covariances, in a parametric, maximum likelihood setting, has additionally been investigated by \cite{Tsiligkaridis2012b} and \cite{Tsiligkaridis2012}, while a much simpler approach of adding an L1 penalty term, for both matrices, to the log-likelihood has been presented in\cite{Zhang2010a}.

%\fixme{cite Glasso}

While the use of matrix-variate priors has been motivated here as a means of facilitating multi-task learning, such methods can also be used as a way of factorising a large matrix into two considerably smaller ones. (from $(LF)^2$ to $L^2 + F^2$) enabled by the Kronecker product structure.

For example, in the case of the matrix completion problem where rows or columns can be treated as independent features, a natural distribution over the matrix is as matrix-normal with separate covariances. As there is only one matrix, $X \in \MReal{D}{P}$ one cannot reliably estimate the mean matrix, but if the mean matrix where itself factored into row and column parts, one would arrive at the model\cite{Allen2010}  $X \sim \mnor{\vv{m}_c \one_P\T + \vv{m}_r \one_N\T}{\Omega}{\Sigma}$ where $\vv{m}_c \in \VReal{D}$ is the mean of the columns, and $\vv{m}_r \in \VReal{P}$ is the mean of the rows. This can equally be explained as a random-effects model with $X_{dp} = m_{c,d} + m_{r,p} + \epsilon_{dp}$ where $\epsilon_{dp} \sim \nor{0}{\Sigma_{dd}\Omega_{pp}}$. Unfortunately, the fact that typical recommender datasets features large number of both users and products limits the applicability of this approach unless one simultaneously reduces the rank of \emph{both} covariances.


%A non-parametric extension of this model was presented in\cite{Yu2009} where $y_{nl} = m_{nl} + f_{nl} + \epsilon_{nl}$, $m_{nl}$ is a function of the task and input features, $f_{nl}$ are the random effects and $\epsilon_{nl}$ is IID zero-mean Gaussian noise. A matrix-variate GP was employed for $m_{nl} \sim \gp{0}{\Omega_0 \otimes \Sigma}$ while $L$ multi-variate GPs were employed for the random effects $f_{l} \sim \gp{0}{\tau \Sigma}$. An inverse Wishart process\cite{Dawid1981} was used as a prior for $\Sigma \sim \mathcal{IWP} \left(\kappa, \Sigma_0 + \lambda\delta\right)$. $\Omega_0$ and $\Sigma_0$ were the covariance functions over task and input features respectively. The marginal distribution over Y is shown to be a matrix-variate Student-T process


\bibliographystyle{plain}
\bibliography{/Users/bryanfeeney/Documents/library.bib}

\end{document}


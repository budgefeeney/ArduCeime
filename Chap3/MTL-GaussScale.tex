\documentclass[10pt,fleqn]{article}

\usepackage[english]{babel}
\usepackage[utf8x]{inputenc}
\usepackage{enumerate}
\usepackage{amsmath}
\usepackage{amssymb}
\usepackage{amsfonts} 
\usepackage{mathtools}
\usepackage{graphicx}
\usepackage{bm}
\usepackage[usenames,dvipsnames]{color}
\usepackage{todonotes}
\usepackage{dsfont}
\usepackage{hyperref}
\hypersetup{
    colorlinks,
    citecolor=black,
    filecolor=black,
    linkcolor=black,
    urlcolor=black
}
\usepackage{algorithm}
\usepackage{algorithmic}
\usepackage{appendix}
\usepackage{subcaption}
\usepackage{fancyvrb}
\usepackage{subfigure}
\usepackage{graphicx,xcolor}
\usepackage{pifont,mdframed}
\usepackage{tikz}
\usepackage{bm}
\usetikzlibrary{fit,positioning}


%
% Macros
%
\newcommand \cashort[1] { {\todo[color=yello]{#1 -- Cedric}} }
\newcommand \calong[1]  { { \todo[inline,color=yellow]{#1 -- Cedric} } }
\newcommand \gbshort[1] { {\todo[color=cyan!40]{#1 -- Guillaume}} }
\newcommand \gblong[1]  { { \todo[inline, color=cyan!40]{#1 -- Guillaume} } }
\newcommand \mgshort[1] { {\todo{#1 -- Mark}} }
\newcommand \mglong[1]  { { \todo[inline]{#1 -- Mark} } }
\newcommand \bfshort[1] { {\todo[color=green!40]{#1 -- Bryan}} }
\newcommand \bflong[1]  { { \todo[inline,color=green!40]{#1 -- Bryan} } }


% Adds a plus const to the end of a math expression
\def \pcst{+\text{const}}

% A fancy version for capital R
\def \Rcal{\mathcal{R}}

% A fancy version for r
\def \rcal{\mathbf{r}}

% Loss function / log likelihood as appropriate
\def \L{\mathcal{L}}

% KL divergence [Math Mode]
\newcommand{\kl}[2] {
	\text{KL}\left[#1||#2\right]
}

\newcommand \vecf[1] {
    \text{vec}\left(#1\right)
}

\newcommand \ent[1] {
    \text{H} \left[ #1 \right]
}

\newcommand \mut[2] {
    \text{I} \left[ #1 ; #2 \right]
}

\newcommand \dvi[2] {
    \text{D}_\text{VI} \left[ #1; #2 \right]
}

% Starts an expected value expresses [Math Mode]
\newcommand{\starte}[1] {%
	\mathbb{E}_{#1}\left[
}

% Ends an expected value expression [Math Mode]
\def \ende{\right]}

% Starts an varianc expresses [Math Mode]
\newcommand{\startv}[1] {%
	\mathbb{V}\text{ar}_{#1}\left[
}

% Ends an variance expression [Math Mode]
\def \endv{\right]}

%\newcommand \ex[2] {
%    \bigl\langle #1 \bigr\rangle_{#2}
%}
\newcommand \ex[2] {
    \mathbb{E}_{ { #2 } }\left[ #1 \right]
}
\newcommand \var[2] {
    \mathbb{V}ar_{ { #2 } }\left[ #1 \right]
}

\newcommand \halve[1] {
	\frac{#1}{2}
}

\newcommand \half {
    \halve{1}
}

\newcommand \tr { \text{tr} } 

\newcommand \T { ^\top } 

\newcommand \fixme[1] {
    {\color{red} FIXME: #1}
}

\newcommand \vv[1] { \bm #1 }

\newcommand{\mbeq}{\overset{!}{=}}

\newcommand \diag[1] { \text{diag} \left( {#1} \right) }
\newcommand \diagonal[1] { \text{diagonal} \left( {#1} \right) }

\newcommand \Ed {{ \vv{\xi}_d}}
\newcommand \Edj {{\xi_{dj}}}
\newcommand \Edk {{\xi_{dk}}}
\newcommand \AEdj {{\Lambda(\xi_{dj})}}
\newcommand \AEdk {{\Lambda(\xi_{dk})}}
\newcommand \AEd  {{ \bm{\Lambda}(\bm{\xi}_d) }}

\newcommand \Axi { { \Lambda_{\xi} } }
\newcommand \bxi { { \vv{b}_{\xi} } }
\newcommand \cxi { { c_{\xi} } }


\newcommand \wdoc      { { \vv{w}_d } }
\newcommand \wdt[0]  { { w_{dt} } }
\newcommand \wdn[0]  { { \vv{w}_{dn} } }
\newcommand \wdnt[0]  { { w_{dnt} } }
\newcommand \wdd[0]   { { \vv w_{d} } }
\newcommand \zd[0]   { { \vv z_{d} } }
\newcommand \zdn[0]  { { \vv{z}_{dn} } }
\newcommand \zdnk[0] { { z_{dnk} } }
\newcommand \zdk[0]  { { z_{dk} } }
\newcommand \thd[0]  { { \vv \theta_d } }
\newcommand \thdk[0] { { \theta_{dk} } }
\newcommand \thdj[0] { { \theta_{dj} } }
\newcommand \epow[1] { { e^{#1} } }
\newcommand \pkt     { { \phi_{kt}  } }
\newcommand \pk      { { \vv \phi_k } }
\newcommand \lmd     { { \vv \lambda_d } }
\newcommand \lmdk    { { \lambda_{dk} } }
\newcommand \xd      { { \vv x_d } }
\newcommand \atxd     { A ^\top \bm x_d}
\newcommand \axd     { A\bm x_d}
\newcommand \tsq      { { \tau^2 } }
\newcommand \ssq      { { \sigma^2 } }
\newcommand \tmsq     { { \tau^{-2} } }
\newcommand \asq      { { \alpha^2 } }
\newcommand \amsq     { { \alpha^{-2} } }
\newcommand \sgsq     { { \sigma^2 } }
\newcommand \xvec     { { \vv{x} } }
\newcommand \omk      { { \bm \omega _k } }
\newcommand \omkt     { { \omega_{kt} } }
\newcommand \oma     { { \Omega_A } }
\newcommand \gdn      { { \vv{\gamma}_{dn} } }
\newcommand \gdnk     { { \gamma_{dnk} } }
\newcommand \gdk      { { \gamma_{dk} } }
\newcommand \isigt   { { \Sigma^{-1}_{\bm \theta} } }




\newcommand \halfSig { \frac{1}{2\sigma^2} }

\newcommand \nor[2]   { \mathcal{N} \left( {#1}, {#2} \right) }
\newcommand \nord[3]   { \mathcal{N}_{#1} \left( {#2}, {#3} \right) }
\newcommand \mnor[3]  { \mathcal{N} \left(#1, #2, #3\right) }
\newcommand \norp[3]  { \mathcal{N} \left(#1; #2, #3\right) }
\newcommand \mnorp[4] { \mathcal{N} \left(#1; #2, #3, #4\right) }
\newcommand \mul[1]   { \mathcal{M} \left( {#1} \right) }
\newcommand \muln[2]  { \mathcal{M} \left( {#1},{#2} \right) }
\newcommand \dir[1]   { \mathcal{D} \left( {#1} \right) }
\newcommand \pois[1]  { \mathcal{P} \left( {#1} \right) }
\newcommand \gp[2]    { \mathcal{GP} \left( {#1}, #2 \right) }
\newcommand \dir[1]   { \mathcal{D} \left( {#1} \right) }
\newcommand \gam[2]   { \mathcal{G} \left( {#1}, {#2} \right) }
\newcommand \beta[1]  { \mathcal{B}eta \left( {#1}, {#2} \right) }

\newcommand \lne[1]  { { \ln \left( 1 + e^{ #1 } \right) } }
\newcommand \Tr[1]   { \tr \left(  {#1}  \right) }

\newcommand \roud  { \vv{\rho}_{d}  }
\newcommand \rodk { \rho_{dk} }

\newcommand \exA[1]  { \ex{#1}{q(A)} }
\newcommand \exV[1]  { \ex{#1}{q(V)} }
\newcommand \exT[1]  { \ex{#1}{q(\Theta)} }
\newcommand \extd[1] { \ex{#1}{q(\thd)} }
\newcommand \exTV[1] { \ex{#1}{q(\Theta)q(V)} }

\newcommand \Real[0]  { { \mathbb{R} } }
\newcommand \VReal[1] { { \mathbb{R}^{#1} } }
\newcommand \MReal[2] { { \mathbb{R}^{#1 \times #2} } }
\newcommand \Nat[0]  { { \mathbb{N} } }
\newcommand \VNat[1] { { \mathbb{N}^{#1} } }
\newcommand \MNat[2] { { \mathbb{N}^{#1 \times #2} } }

\newcommand \inv[1] { {#1}^{-1} }
\newcommand \invb[1] { \inv{\left( #1 \right)} }

\newcommand \cn { \textsuperscript{\texttt{[{\color{blue}Citation Needed}]}} }

\newcommand \const { { \text{c} } }

\providecommand \floor [1] { \left \lfloor #1 \right \rfloor }
\providecommand \ceil [1] { \left \lceil #1 \right \rceil }


\newcommand \vt[2] { { #1^{(#2)} } }

\newcommand \hashtag[1] { { \ttfamily \##1 } }

\newcommand \mvy  { \vv{m}_{\vv{y}} }
\newcommand \sigvy { { S_Y } }

\newcommand \mmy  { M_Y      }
\newcommand \md   { \vv{m}_d }
\newcommand \phin { \vv{\phi}_n }
\newcommand \isigma { { \inv{\Sigma} } }

\newcommand \sigv     { { \Sigma_V } }
\newcommand \isigv     { { \Sigma^{-1}_V } }

\newcommand \sigy { { \Sigma_Y } }
\newcommand \isigy { { \Sigma_{-1}_Y } }


\newcommand \omy  { { \Omega_Y } }
\newcommand \iomy { { \inv{\Omega_Y} } }

\newcommand \siga     { { \Sigma_A } }
\newcommand \isiga     { { \Sigma^{-1}_A } }
\newcommand \diagv { { \diag{\nu_1,\ldots,\nu_P} } }

\newcommand \ma { \vv{m}_a }
\newcommand \my { \vv{m}_y }

\newcommand \VoU { V \otimes U }

\newcommand \one { \mathbb{1} }
%\newcommand \one  {{  \mathds{1} }}

\newcommand \lse { \text{lse} }
%\newcommand \lse[0] { \mathrm{lse} }

% Conditional independence 
\def\ci{\perp\!\!\!\perp} % from Wikipedia



% ------ For the eval section

% Multinomial PDF [Math Mode]
% params: 1 - the variable
%         2 - the value
%         3 - the state indicator (e.g. k for a distro with K values)
%         4 - any additional subscript
\newcommand{\mpdf}[4] {
	\prod_{#3} {#1}_{{#4} {#3}} ^ {#2}
}

% Dirichlet PDF [Math Mode]
% params: 1 - the variable
%         2 - the hyper-parameter
%         3 - the state indicator (e.g. k for a distro with K values)
%         4 - any additional subscript
\newcommand{\dpdf}[4] {
	\frac{1}{B({#2})} \prod_{#3} {#1}_{{#4} {#3}} ^ {({#2}_{#3} - 1)}
}

% To simplify the sampling equations, this is indicates
% that the given value has had datapoint "m" stripped out
%
\newcommand{\lm}[1] {
	#1^{\setminus m}
}

\newcommand \model[0] {
    \mathcal{M}
}

\newcommand \perplexity[1] {
    \mathcal{P} \left( { #1 } \right)
}

\newcommand \WTrain {
    \mathcal{W}^{(t)}
}

\newcommand \WQuery {
    \mathcal{W}^{(q)}
}

\newcommand \oneover[1] {
    \frac{1}{ {#1} }
}

\newcommand \samp[1] {
    { #1 }^{(s)}
}

\newcommand \etd[0] {
    \vv{\eta}_d
}

\begin{document}



\section{Future Work}
\subsection{Sparsity Inducing Priors}
In the two examples given we've used sparse, binary features. Low rank covariance decompositions may not be a good fit for such methods.

With that in mind we may instead choose to sparsify the features, by learning a sparse precision matrix.

In the previous section on multi-task learning we described how using either L1 regularisation or a Laplace prior on the a vectorized precision matrix will lead to a preference for sparse solutions. An alternate approach, which is slightly more robust, and encompasses a broader range of distributions is the use of a Gaussian Scale mixture.

For example, if one places a Gaussian-Exponential prior on each dimension of a weight vectors $\vv{w}_f \sim \nor{0}{\alpha_f}$ and places an exponential prior on the variance $\tau_f \sim \mathcal{E}\left(\gamma\right)$ then the marginal distribution over $w_f$ is a zero-mean Laplace distribution, where $\gamma$ controls the degree of sparsity \cite{Figueiredo2003}.

Similarly, if one employs a Gaussian-Gamma prior instead, the resulting distribution is a Student-T distribution, but now in a form amendable to inference using EM where with a latent variable $u$

\begin{align}
u & \sim \mathcal{G}\left(\halve{\nu}, \halve{\nu} \right) ,
& \vv{z}|u \sim \nor{\vv{\mu}}{\nu \inv{\Sigma}} 
& \implies & \vv{z} & \sim \mathcal{S}\left(\vv{\mu}, \inv{\Sigma}, \nu \right)
\end{align}

As the Student-T distribution has ``heavier" tails than the Gaussian, it is more accommodating of outliers, and so replacing Gaussian likelihood with Student-T likelihoods can lead algorithms which are more ``robust" to outliers, such as PCA and CCA\cite{Archambeau2006a}, and also matrix factorization\cite{Balaji2011}.

Likewise if one places a different Gaussian-Gamma mix on every individual component of a matrix $W \in \MReal{L}{F}$ of the form
\begin{align}
w_{ij} &\sim \nor{0}{\alpha_{ij}} & \alpha_{ij} \sim \mathcal{G}\left(\frac{a}{LF}, b\right)
\end{align}
one obtains a sparsifying inference scheme\cite{Archambeau2009a} similar automatic relevance-determination (ARD)

Such a model does not directly correspond to any known distribution. If one sets $a=0$ the marginal over $w_{ij}$ is the Laplace distribution, while the marginal approaches a Normal-Jeffry's prior as both $a$ and $b$ approach zero. Finally, if $\frac{a}{LF} = b$ then one recovers a product of student-T distributions, which is also sharply peaked around zero and so sparsity inducing. The posterior distribution over the $\alpha_{ij}$ is a produce of generalised inverse Gaussian distributions.

This permits an alternative, and more robust,  approach to sparse, correlated multi-task regression than that described earlier using the Glasso approach. Placing a matrix-variate prior over the multi-task weights matrix, and then placing a Gaussian-Gamma prior on the covariance over features, allows for inference of sparse weights. However, there are a few extensions. First, if the features have a block structure (e.g. multiple 1-of-F categorical encodings), one can place priors over weights corresponding to \emph{blocks} of features $W_i$. At the same time, if there are many tasks, one can learn a \emph{low-rank} distribution over tasks, using the the same derivation of as PCA. The resulting model\cite{Archambeau2011} is:

\begin{align}
\vv{t}_n|W,\vv{x}_n & \sim \nor{W\vv{x}_n}{\sigma^2 I_L} &
V & \sim \mnor{0}{\tau I_L}{I_Q} \\
W_i | V, Z_i, \Omega_i,\alpha_i & \sim \mnor{VZ_i}{\alpha_i^{-1} \Omega_i}{I_L} &
\Omega_i & \sim \mathcal{W}^{-1}\left(\nu, \lambda I_F \right) \\
Z_i | \Omega_i & \sim \mnor{0}{\alpha_i^{-1} \Omega_i}{I_Q} &
\alpha_i & \sim\mathcal{N}^{-1}\left(\omega, \chi, \phi\right)
\end{align}

Where $Q \ll L$ is the size of the low-rank task space, and $\mathcal{N}^{-1}\left(\omega, \chi, \phi\right)$ denotes the generalized inverse Gaussian distribution with PDF
\begin{align}
p(\alpha_i) = \frac{\chi^{-\omega}\left(\sqrt{\chi \phi}\right)^\omega}{2 K_w \left(\sqrt{\chi \phi}\right)} \alpha_i^{\omega-1} e^{-\half \chi \alpha_i^{-1} + \phi \alpha_i}
\end{align}
with $K_w(\cdot)$ being a modified Bessel function of the second kind. Inference, using variational EM, is complex, with gradient based methods required to learn the parameters of the GIG distribution.

The use of a generalised inverse Gaussian (GIG) distribution as a prior over \emph{prior} Gaussian-Scale mixture means that the marginal distribution over the weight matrix -- depending on its inferred parameterisation -- can be a matrix-variate Laplace, Student-T or Gamma distribution. The identifiability issues caused by the Kronecker matrix are handled by the shrinkage priors.

This model can be extended\cite{Yang2011} by defining a \emph{matrix-variate} generalised inverse Gaussian (MGIG) distribution $G\sim \mathcal{MGIG}\left(\Psi, \Phi, \nu\right), G \in \MReal{F}{F}$
\begin{align}
\frac{|G|^{\omega - (F + 1)/2}}{|\half \Psi|^\omega K_\omega \left(\frac{1}{4}\Phi \Psi\right)} \text{etr}\left( -\half \Psi \inv{G} - \half \Phi G \right)
\end{align}
where $K_\omega(\cdot)$ is a matrix-variate Bessel function. The Wishart and inverse-Wishart distributions are a special case of the MGIG distribution. Combined with a Gaussian-scale mixture this leads to a Gaussian MGIG (GMGIG) prior over weights.
\begin{align}
W & \sim \mnor{VZ}{\Omega}{\Sigma} &
V & \sim \mnor{V_0}{\kappa_v I_P}{\Sigma} &
Z & \sim \mnor{Z_0}{\Omega}{\kappa_z I_P} \\
\Omega & \sim \mathcal{MGIG}\left( \Psi_\Omega, \Phi_\Omega, \nu_\Omega \right) &
\Sigma & \sim \mathcal{MGIG}\left( \Psi_\Sigma, \Phi_\Sigma, \nu_\Sigma \right) &
\end{align}
which as can be seen, is a prior that assumes W is of low rank, with latent features $P \ll F$. Like the GIG prior of this, depending on the parameterisation, encompasses matrix-variate Student-T, Laplace and Bessel distributions. 

With this in mind, the model could be transformed to use a sparse prior on the features covariance by incorporating a Gauss-Scale mixture prior on the features covariance.

%\fixme{This paper notes you can add a condition that $\tr{\Epsilon} \leq 1$ to avoid it going to infinity, which would have been good to know.}


\bibliographystyle{plain}
\bibliography{/Users/bryanfeeney/Documents/library.bib}

\end{document}



\documentclass[10pt,fleqn]{article}

\usepackage[english]{babel}
\usepackage[utf8x]{inputenc}
\usepackage{enumerate}
\usepackage{amsmath}
\usepackage{amssymb}
\usepackage{amsfonts} 
\usepackage{mathtools}
\usepackage{graphicx}
\usepackage{bm}
\usepackage[usenames,dvipsnames]{color}
\usepackage{todonotes}
\usepackage{dsfont}
\usepackage{hyperref}
\hypersetup{
    colorlinks,
    citecolor=black,
    filecolor=black,
    linkcolor=black,
    urlcolor=black
}
\usepackage{algorithm}
\usepackage{algorithmic}
\usepackage{appendix}
\usepackage{subcaption}
\usepackage{fancyvrb}
\usepackage{subfigure}
\usepackage{graphicx,xcolor}
\usepackage{pifont,mdframed}
\usepackage{tikz}
\usepackage{bm}
\usetikzlibrary{fit,positioning}


%
% Macros
%
\newcommand \cashort[1] { {\todo[color=yello]{#1 -- Cedric}} }
\newcommand \calong[1]  { { \todo[inline,color=yellow]{#1 -- Cedric} } }
\newcommand \gbshort[1] { {\todo[color=cyan!40]{#1 -- Guillaume}} }
\newcommand \gblong[1]  { { \todo[inline, color=cyan!40]{#1 -- Guillaume} } }
\newcommand \mgshort[1] { {\todo{#1 -- Mark}} }
\newcommand \mglong[1]  { { \todo[inline]{#1 -- Mark} } }
\newcommand \bfshort[1] { {\todo[color=green!40]{#1 -- Bryan}} }
\newcommand \bflong[1]  { { \todo[inline,color=green!40]{#1 -- Bryan} } }


% Adds a plus const to the end of a math expression
\def \pcst{+\text{const}}

% A fancy version for capital R
\def \Rcal{\mathcal{R}}

% A fancy version for r
\def \rcal{\mathbf{r}}

% Loss function / log likelihood as appropriate
\def \L{\mathcal{L}}

% KL divergence [Math Mode]
\newcommand{\kl}[2] {
	\text{KL}\left[#1||#2\right]
}

\newcommand \vecf[1] {
    \text{vec}\left(#1\right)
}

\newcommand \ent[1] {
    \text{H} \left[ #1 \right]
}

\newcommand \mut[2] {
    \text{I} \left[ #1 ; #2 \right]
}

\newcommand \dvi[2] {
    \text{D}_\text{VI} \left[ #1; #2 \right]
}

% Starts an expected value expresses [Math Mode]
\newcommand{\starte}[1] {%
	\mathbb{E}_{#1}\left[
}

% Ends an expected value expression [Math Mode]
\def \ende{\right]}

% Starts an varianc expresses [Math Mode]
\newcommand{\startv}[1] {%
	\mathbb{V}\text{ar}_{#1}\left[
}

% Ends an variance expression [Math Mode]
\def \endv{\right]}

%\newcommand \ex[2] {
%    \bigl\langle #1 \bigr\rangle_{#2}
%}
\newcommand \ex[2] {
    \mathbb{E}_{ { #2 } }\left[ #1 \right]
}
\newcommand \var[2] {
    \mathbb{V}ar_{ { #2 } }\left[ #1 \right]
}

\newcommand \halve[1] {
	\frac{#1}{2}
}

\newcommand \half {
    \halve{1}
}

\newcommand \tr { \text{tr} } 

\newcommand \T { ^\top } 

\newcommand \fixme[1] {
    {\color{red} FIXME: #1}
}

\newcommand \vv[1] { \bm #1 }

\newcommand{\mbeq}{\overset{!}{=}}

\newcommand \diag[1] { \text{diag} \left( {#1} \right) }
\newcommand \diagonal[1] { \text{diagonal} \left( {#1} \right) }

\newcommand \Ed {{ \vv{\xi}_d}}
\newcommand \Edj {{\xi_{dj}}}
\newcommand \Edk {{\xi_{dk}}}
\newcommand \AEdj {{\Lambda(\xi_{dj})}}
\newcommand \AEdk {{\Lambda(\xi_{dk})}}
\newcommand \AEd  {{ \bm{\Lambda}(\bm{\xi}_d) }}

\newcommand \Axi { { \Lambda_{\xi} } }
\newcommand \bxi { { \vv{b}_{\xi} } }
\newcommand \cxi { { c_{\xi} } }


\newcommand \wdoc      { { \vv{w}_d } }
\newcommand \wdt[0]  { { w_{dt} } }
\newcommand \wdn[0]  { { \vv{w}_{dn} } }
\newcommand \wdnt[0]  { { w_{dnt} } }
\newcommand \wdd[0]   { { \vv w_{d} } }
\newcommand \zd[0]   { { \vv z_{d} } }
\newcommand \zdn[0]  { { \vv{z}_{dn} } }
\newcommand \zdnk[0] { { z_{dnk} } }
\newcommand \zdk[0]  { { z_{dk} } }
\newcommand \thd[0]  { { \vv \theta_d } }
\newcommand \thdk[0] { { \theta_{dk} } }
\newcommand \thdj[0] { { \theta_{dj} } }
\newcommand \epow[1] { { e^{#1} } }
\newcommand \pkt     { { \phi_{kt}  } }
\newcommand \pk      { { \vv \phi_k } }
\newcommand \lmd     { { \vv \lambda_d } }
\newcommand \lmdk    { { \lambda_{dk} } }
\newcommand \xd      { { \vv x_d } }
\newcommand \atxd     { A ^\top \bm x_d}
\newcommand \axd     { A\bm x_d}
\newcommand \tsq      { { \tau^2 } }
\newcommand \ssq      { { \sigma^2 } }
\newcommand \tmsq     { { \tau^{-2} } }
\newcommand \asq      { { \alpha^2 } }
\newcommand \amsq     { { \alpha^{-2} } }
\newcommand \sgsq     { { \sigma^2 } }
\newcommand \xvec     { { \vv{x} } }
\newcommand \omk      { { \bm \omega _k } }
\newcommand \omkt     { { \omega_{kt} } }
\newcommand \oma     { { \Omega_A } }
\newcommand \gdn      { { \vv{\gamma}_{dn} } }
\newcommand \gdnk     { { \gamma_{dnk} } }
\newcommand \gdk      { { \gamma_{dk} } }
\newcommand \isigt   { { \Sigma^{-1}_{\bm \theta} } }




\newcommand \halfSig { \frac{1}{2\sigma^2} }

\newcommand \nor[2]   { \mathcal{N} \left( {#1}, {#2} \right) }
\newcommand \nord[3]   { \mathcal{N}_{#1} \left( {#2}, {#3} \right) }
\newcommand \mnor[3]  { \mathcal{N} \left(#1, #2, #3\right) }
\newcommand \norp[3]  { \mathcal{N} \left(#1; #2, #3\right) }
\newcommand \mnorp[4] { \mathcal{N} \left(#1; #2, #3, #4\right) }
\newcommand \mul[1]   { \mathcal{M} \left( {#1} \right) }
\newcommand \muln[2]  { \mathcal{M} \left( {#1},{#2} \right) }
\newcommand \dir[1]   { \mathcal{D} \left( {#1} \right) }
\newcommand \pois[1]  { \mathcal{P} \left( {#1} \right) }
\newcommand \gp[2]    { \mathcal{GP} \left( {#1}, #2 \right) }
\newcommand \dir[1]   { \mathcal{D} \left( {#1} \right) }
\newcommand \gam[2]   { \mathcal{G} \left( {#1}, {#2} \right) }
\newcommand \beta[1]  { \mathcal{B}eta \left( {#1}, {#2} \right) }

\newcommand \lne[1]  { { \ln \left( 1 + e^{ #1 } \right) } }
\newcommand \Tr[1]   { \tr \left(  {#1}  \right) }

\newcommand \roud  { \vv{\rho}_{d}  }
\newcommand \rodk { \rho_{dk} }

\newcommand \exA[1]  { \ex{#1}{q(A)} }
\newcommand \exV[1]  { \ex{#1}{q(V)} }
\newcommand \exT[1]  { \ex{#1}{q(\Theta)} }
\newcommand \extd[1] { \ex{#1}{q(\thd)} }
\newcommand \exTV[1] { \ex{#1}{q(\Theta)q(V)} }

\newcommand \Real[0]  { { \mathbb{R} } }
\newcommand \VReal[1] { { \mathbb{R}^{#1} } }
\newcommand \MReal[2] { { \mathbb{R}^{#1 \times #2} } }
\newcommand \Nat[0]  { { \mathbb{N} } }
\newcommand \VNat[1] { { \mathbb{N}^{#1} } }
\newcommand \MNat[2] { { \mathbb{N}^{#1 \times #2} } }

\newcommand \inv[1] { {#1}^{-1} }
\newcommand \invb[1] { \inv{\left( #1 \right)} }

\newcommand \cn { \textsuperscript{\texttt{[{\color{blue}Citation Needed}]}} }

\newcommand \const { { \text{c} } }

\providecommand \floor [1] { \left \lfloor #1 \right \rfloor }
\providecommand \ceil [1] { \left \lceil #1 \right \rceil }


\newcommand \vt[2] { { #1^{(#2)} } }

\newcommand \hashtag[1] { { \ttfamily \##1 } }

\newcommand \mvy  { \vv{m}_{\vv{y}} }
\newcommand \sigvy { { S_Y } }

\newcommand \mmy  { M_Y      }
\newcommand \md   { \vv{m}_d }
\newcommand \phin { \vv{\phi}_n }
\newcommand \isigma { { \inv{\Sigma} } }

\newcommand \sigv     { { \Sigma_V } }
\newcommand \isigv     { { \Sigma^{-1}_V } }

\newcommand \sigy { { \Sigma_Y } }
\newcommand \isigy { { \Sigma_{-1}_Y } }


\newcommand \omy  { { \Omega_Y } }
\newcommand \iomy { { \inv{\Omega_Y} } }

\newcommand \siga     { { \Sigma_A } }
\newcommand \isiga     { { \Sigma^{-1}_A } }
\newcommand \diagv { { \diag{\nu_1,\ldots,\nu_P} } }

\newcommand \ma { \vv{m}_a }
\newcommand \my { \vv{m}_y }

\newcommand \VoU { V \otimes U }

\newcommand \one { \mathbb{1} }
%\newcommand \one  {{  \mathds{1} }}

\newcommand \lse { \text{lse} }
%\newcommand \lse[0] { \mathrm{lse} }

% Conditional independence 
\def\ci{\perp\!\!\!\perp} % from Wikipedia



% ------ For the eval section

% Multinomial PDF [Math Mode]
% params: 1 - the variable
%         2 - the value
%         3 - the state indicator (e.g. k for a distro with K values)
%         4 - any additional subscript
\newcommand{\mpdf}[4] {
	\prod_{#3} {#1}_{{#4} {#3}} ^ {#2}
}

% Dirichlet PDF [Math Mode]
% params: 1 - the variable
%         2 - the hyper-parameter
%         3 - the state indicator (e.g. k for a distro with K values)
%         4 - any additional subscript
\newcommand{\dpdf}[4] {
	\frac{1}{B({#2})} \prod_{#3} {#1}_{{#4} {#3}} ^ {({#2}_{#3} - 1)}
}

% To simplify the sampling equations, this is indicates
% that the given value has had datapoint "m" stripped out
%
\newcommand{\lm}[1] {
	#1^{\setminus m}
}

\newcommand \model[0] {
    \mathcal{M}
}

\newcommand \perplexity[1] {
    \mathcal{P} \left( { #1 } \right)
}

\newcommand \WTrain {
    \mathcal{W}^{(t)}
}

\newcommand \WQuery {
    \mathcal{W}^{(q)}
}

\newcommand \oneover[1] {
    \frac{1}{ {#1} }
}

\newcommand \samp[1] {
    { #1 }^{(s)}
}

\newcommand \etd[0] {
    \vv{\eta}_d
}

\begin{document}



\newcommand \ld { \vv{l}_d }
\newcommand \thod {{ \vv{\theta}_{\cdot d} }}
\newcommand \thko {{ \vv{\theta}_{k \cdot} }}

\subsection{Incorporating Downstream Features}
The division of ``upstream" and ``downstream" models is somewhat artificial, as it is entirely possible to combine the approaches to derive a model predicting $p(\wdoc, \ld | \xd) = \int p(\wdoc, \ld|\thd,\Phi) p(\thd | \xd) d\thd d\Phi$.

One such example is a model for predicting both the words and citations (or ``out-links") of academic papers given certain metadata, such as author, journal or year.

Many approaches involve learning the topics for the words of documetns in the corpus, and then predicting the existence of an (undirected) link between two documents as a function of their topic-distributions. In the relational topic-model\cite{Chang2009a}\cite{Chang2010a}, topics are inferred broadly as for LDA, however, the probability of a links between documents $d$ and $p$ is predicted as

\begin{align*}
y_{dp} & = \sigma(\vv{\eta}\T (\bar{\vv{z}_d} .* \bar{\vv{z}_p}) + \nu) &
\bar{\vv{z}_d} & = \frac{1}{N_d} \vv{z}_{dn}
\end{align*}
This utility is a weighted distance metric, where the extra factor $\nu$ is added to account for the fact that link structure doesn't always match document content\footnote{Documents typically contain substantial content about prior work, but are typically cited due to their contribution only.}. The best results are given when the activation function is the probit. This in turn creates a dependency of the posterior $q(z_{dn})$ on links:

\begin{align*}
q(z_{dn}) \propto \exp\left(
\mathbb{E}_q\left[ \ln \theta_d \right]
+ \mathbb{E}_q\left[ \ln \beta{\cdot w_{dn}} \right]
+ \sum_{p \neq d} \nabla_{\ex{z_p}{q}} \mathcal{F}(q)
\right)
\end{align*}
where $\mathcal{F}(q)$ is the bound on the log-likelihood.

This is very similar to the supervised LDA model \cite{Blei2008} and a variant of this model was additionally published in\cite{Neiswanger2014}. In our experiments however, the ability of RTM to rank papers adequately (e.g. as a recommender) has been poor.

A different approach to the problem of link prediction where the content of all papers is available is the LDA/PLSA model\cite{Nallapati2008}\cite{Nallapati2008a} in that the topics used to generate links differ to those used to generate words, though both share the same Dirichlet prior. In the LDA/PLSA model, the $K \times L$ matrix distribution of per-topic link-distributions used in LDA can be used as $L \times K$ matrix of per-link topic-distributions used in PLSA to model the linked documents' words. Sharing the parameter allows a transfer of learning from the words of the linked documents to the generation of links in linking documents.

Other approaches to link prediction have ignored the document content and focussed on network structure instead, modelling the network as a partially observed admixture of network cliques. Known as mixed-membership stochastic block-models, the general approach\cite{Blei2009} considers all edges to be undirected, and assume that each edge has a topic from the originating and destination nodes.

\begin{align*}
\thd & \sim \dir{\vv{\alpha}} &
z_{dp} & \sim \muln{\vv{\thd}}{1} &
z_{pd} & \sim \muln{\vv{\theta}_p}{1} \\
y_{dp} & = \epsilon + \sum_k z^k_{dp} z^k_{pd} \left( \beta_k - \epsilon\right) 
\end{align*}

Variants of this have incorporated node popularity for each node\cite{Gopalan2013a} such that $y_{dp} = \pi_d + \pi_p + \sum_k z^k_{dp} z^k_{pd} \beta_k$, where $\pi_d \sim \nor{0}{\alpha^2}$ is the $d$-th node's popularity, and the $\beta_k$ is the community strength, a beta-distributed measure of how densely nodes within community $k$ are linked.

Link prediction is an unbalanced prediction problem, with very few positive examples in the dataset. In the RTM model\cite{Chang2009a}\cite{Chang2010a} a single-class classifier was trained, regularised by using a combination of Gaussian priors and the addition of a number of pseudo-documents for the negative case drawn from the posterior, their number indicated by means of a parameter $\rho$. In the case of stochastic block-models, weighted stochastic graident ascent methods have been used, notably in \cite{Gopalan2013a}\cite{Gopalan2013b}.

With this in mind, one could derive a self-referential topic model, extending the idea of LDA-PLDA, where the $K \times D$ matrix of topic distributions is given by: 

\begin{equation}
\ln p(\Theta) = -\halve{DK}\ln 2\pi 
- \halve{K}\ln |\Omega|
- \halve{D}\ln |\Sigma|
- \half \Tr{\inv{\Sigma}(\Theta - M)\inv{\Omega}(\Theta - M)\T}
\end{equation}

We can use this to generate topics for every possible document, and then slice it column-wise to get a distribution $\thod$ over topics for every document $d$, or row-wise to get a distribution $\thko$ over documents -- i.e. out-links -- for every topic $k$.

As the covariance over documents would be prohibitively large, we use a low-rank approximation, while retaining a full covariance $\Sigma$ over topics.\footnote{I've looked into joint low-rank factorisations before: it's tricky as the result of marginalizing a doubly-transformed latent space $V$ from a matrix-variate distribution $\Theta|V \sim \mnor{QVU\T}{I}{I}$ is not itself a matrix-variate distribution, and so the resulting updates require some unusual transforms like the commutation-matrix and the vec-transpose operator}.


\begin{align}
V & \sim \mnor{0}{I}{I} & 
\Theta|V & \sim \mnor{VU\T + M}{I_D}{\Sigma} & \implies
\Theta & \sim \mnor{M}{I_D + UU\T}{\Sigma}
\end{align}

The generation of words and links then follow
\begin{align}
z_{dn} & \sim \muln{1}{\vv{\sigma}(\thod)} &
w_{dn} & \sim \muln{1}{\pk_{z_{dn}}} &
\pk    & \sim \dir{\vv{\phi}_0} \\
y_{dm} & \sim \muln{1}{\vv{\sigma}(\thod)} &
l_{dm} & \sim \muln{1}{\vv{\sigma}(\vv{\theta}_{y_{dm}\cdot})}
\end{align}

where the logistic sigmoid is $\sigma_k(\thod) = \frac{\exp(\thdk)}{\sum_j \exp(\theta_{dj})}$ and $\vv{\sigma}(\thod)$ is the vector of such values for all $k$.

The updates $Z$, $Y$ and $W$ all follow the standard CTM model. Taking derivatives, the row and column covariances, $R$ and $S$, of the matrix-variate posterior over the topic-document matrix are given by
\begin{align}
\inv{R} & = \Tr{S \inv{\Sigma}}I_D + (N^w + N^l)\Tr{SA^k} + \Tr{S N^i}A^D \label{eqn:post-row-cov}\\
\inv{S} & = \Tr{R I_D}\inv{\Sigma} + \Tr{R (N^w + N^l)}A^K + \Tr{R A^D} N^i \label{eqn:post-col-cov}
\end{align}
while the posterior mean $M_\Theta$ is given by the solution to the following equation
%\begin{align*}
%\begin{split}
%\inv{\Sigma}M_\Theta &+ A^K M_\Theta (N^w + N^l) + N^i M_\Theta A^D  \\
%& = \inv{\Sigma}(UV\T + M) + \hat{Z} - N^w B + \hat{Y} - N^l B + \hat{P}\T - (N^i D)\T
%\end{split}
%\end{align*}

The usual approach to this, a form of the Sylvester equation, is to employ $AXB = (B\T \otimes A)\vecf{X}$, but this involves materialising a prohibitively large $DK \times DK$ matrix. 

\begin{align*}
(\Sigma \otimes I_D & + A^K \otimes (N^w + N^l) + N^i \otimes A^D) \text{ }\vecf{M_\Theta}\\
& = \vecf{\inv{\Sigma}(UV\T + M) + \hat{Z} - N^w B + \hat{Y} - N^l B + \hat{P}\T - (N^i D)\T}
\end{align*}

The non-separability of the covariances indicates that the posterior is not a valid matrix-variate normal distribution.

In the absence of a neat solution, an alternative is to rotate back the traces involved in $\ex{\ln p(L|Y,\Theta)}{q}$ into an equation in terms of $\thod$ instead of $\thko$, and then solve individually for $\thod$ a document at a time. Re-written in these terms, the expected log-probability is:

\begin{align}
\ex{\ln p(L|Y,\Theta)}{q} \propto & \sum_d \hat{P}_{d,\cdot}\T \vv{\theta}_{\cdotd}
-  \sum_d \left(\half \sum_p A^D_{dp} \vv{\theta}_{\cdot d}\T  N^i_{kk} \vv{\theta}_{\cdot p}\right)    
    + D_{\cdot d} \T N^i_{kk} \vv{\theta_{\cdot d}}
\end{align}

In this regime, the update for each $\vv{\theta}_d$ can be calculated as

\begin{align}
\hat{\thd} & = \invb{\inv{\Sigma} + (N^w_{dd} + N^l_{dd})A^K + A^D_{dd} N^i}\\
& \qquad
\left( 
    \inv{\Sigma}(V \vv{u}_d + \vv{m}_d) + Z_d - (N^w_{dd} + N^l_{dd}) \vv{b}_d + P_d - \half \sum_{p \neq d} A^D_{pd} N^i \hat{\vv{\theta}}_p - N^i D_{\cdot, d}
\right) \label{eqn:monster-thd}
\end{align}
where $\vv{u}_d$ is the d-th column of the $P \times D$ matrix U. In this case we can see how row covariance terms $(N^w_{dd} + N^l_{dd})$, $A^D_{dd}$ (and implicitly $I_D$) are being used to modify the column covariances in the posterior covariance.

This fact that the the topic distributions of other documents occur only in a single sum means they need not be stored explicitly, but rather used to modify that sum, making inference tractable, and stochastic variational inference (SVI) possible.

Using quadratic boudns again, the final inference procedure is given in algorithm \ref{alg:sra_generic}.

\begin{algorithm}
\caption{Self-Referential Algorithm}
\label{alg:sra_generic}
    \begin{align*}
        V & = (M_\Theta U - M U)\invb{I_P + U\T U} \\
        U & = \invb{I_D + V V\T}(V\T M_\Theta - V\T M) \\
        S_d & = \invb{\inv{\Sigma} + (N^w_{dd} + N^l_{dd})A^K + A^D_{dd} N^i} \qquad \text{(Not a model parameter)}\\
        \vv{\rho}_d & = \diag{S_d} \\
        \hat{\thd} & = S_d
\left( 
    \inv{\Sigma}(V \vv{u}_d + \vv{m}_d) + Z_d - (N^w_{dd} + N^l_{dd}) \vv{b}_d + P_d - \half \sum_{p \neq d} A^D_{pd} N^i \hat{\vv{\theta}}_p - N^i F_{\cdot, d}
\right) \\
        \hat{z}_{dnk} & \propto \exp\left(\Psi(\hat{\theta}_{dk}) - \Psi(\sum_j \hat{\theta}_{dj}) + \Psi(\hat{\phi}_{kw_{dn}}) - \Psi(\sum_j \hat{\phi}_{jw_{dn}})\right) \\
        \hat{y}_{dmk} & \propto \exp\left(\Psi(\hat{\theta}_{dk}) - \Psi(\sum_j \hat{\theta}_{dj}) + \Psi(\hat{\theta}_{l_{dm}k}) - \Psi(\sum_j \hat{\theta}_{l_{dm}j})\right) \\
        \hat{\phi}_{kt} & = \phi_{0v} + \sum_d \sum_n \hat{z}_{dnk} w_{dnt} \\
        \qquad & \qquad \\
        \Sigma & = \sum_d \diag{\vv{\rho}_d} + (\thod - V \vv{u}_d - \vv{m}_d)(\thod - V \vv{u}_d - \vv{m}_d)\T
    \end{align*}
\end{algorithm}

This could be extended to use upstream features by replacing the prior mean $M$ with a function of the features $W X\T$ where $W \sim \mnor{0}{I_F}{\Sigma}$.



\bibliographystyle{plain}
\bibliography{/Users/bryanfeeney/Documents/library.bib}

\end{document}

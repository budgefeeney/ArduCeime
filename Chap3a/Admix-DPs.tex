\documentclass[10pt,fleqn]{article}

\usepackage[english]{babel}
\usepackage[utf8x]{inputenc}
\usepackage{enumerate}
\usepackage{amsmath}
\usepackage{amssymb}
\usepackage{amsfonts} 
\usepackage{mathtools}
\usepackage{graphicx}
\usepackage{bm}
\usepackage[usenames,dvipsnames]{color}
\usepackage{todonotes}
\usepackage{dsfont}
\usepackage{hyperref}
\hypersetup{
    colorlinks,
    citecolor=black,
    filecolor=black,
    linkcolor=black,
    urlcolor=black
}
\usepackage{algorithm}
\usepackage{algorithmic}
\usepackage{appendix}
\usepackage{subcaption}
\usepackage{fancyvrb}
\usepackage{subfigure}
\usepackage{graphicx,xcolor}
\usepackage{pifont,mdframed}
\usepackage{tikz}
\usepackage{bm}
\usetikzlibrary{fit,positioning}


%
% Macros
%
\newcommand \cashort[1] { {\todo[color=yello]{#1 -- Cedric}} }
\newcommand \calong[1]  { { \todo[inline,color=yellow]{#1 -- Cedric} } }
\newcommand \gbshort[1] { {\todo[color=cyan!40]{#1 -- Guillaume}} }
\newcommand \gblong[1]  { { \todo[inline, color=cyan!40]{#1 -- Guillaume} } }
\newcommand \mgshort[1] { {\todo{#1 -- Mark}} }
\newcommand \mglong[1]  { { \todo[inline]{#1 -- Mark} } }
\newcommand \bfshort[1] { {\todo[color=green!40]{#1 -- Bryan}} }
\newcommand \bflong[1]  { { \todo[inline,color=green!40]{#1 -- Bryan} } }


% Adds a plus const to the end of a math expression
\def \pcst{+\text{const}}

% A fancy version for capital R
\def \Rcal{\mathcal{R}}

% A fancy version for r
\def \rcal{\mathbf{r}}

% Loss function / log likelihood as appropriate
\def \L{\mathcal{L}}

% KL divergence [Math Mode]
\newcommand{\kl}[2] {
	\text{KL}\left[#1||#2\right]
}

\newcommand \vecf[1] {
    \text{vec}\left(#1\right)
}

\newcommand \ent[1] {
    \text{H} \left[ #1 \right]
}

\newcommand \mut[2] {
    \text{I} \left[ #1 ; #2 \right]
}

\newcommand \dvi[2] {
    \text{D}_\text{VI} \left[ #1; #2 \right]
}

% Starts an expected value expresses [Math Mode]
\newcommand{\starte}[1] {%
	\mathbb{E}_{#1}\left[
}

% Ends an expected value expression [Math Mode]
\def \ende{\right]}

% Starts an varianc expresses [Math Mode]
\newcommand{\startv}[1] {%
	\mathbb{V}\text{ar}_{#1}\left[
}

% Ends an variance expression [Math Mode]
\def \endv{\right]}

%\newcommand \ex[2] {
%    \bigl\langle #1 \bigr\rangle_{#2}
%}
\newcommand \ex[2] {
    \mathbb{E}_{ { #2 } }\left[ #1 \right]
}
\newcommand \var[2] {
    \mathbb{V}ar_{ { #2 } }\left[ #1 \right]
}

\newcommand \halve[1] {
	\frac{#1}{2}
}

\newcommand \half {
    \halve{1}
}

\newcommand \tr { \text{tr} } 

\newcommand \T { ^\top } 

\newcommand \fixme[1] {
    {\color{red} FIXME: #1}
}

\newcommand \vv[1] { \bm #1 }

\newcommand{\mbeq}{\overset{!}{=}}

\newcommand \diag[1] { \text{diag} \left( {#1} \right) }
\newcommand \diagonal[1] { \text{diagonal} \left( {#1} \right) }

\newcommand \Ed {{ \vv{\xi}_d}}
\newcommand \Edj {{\xi_{dj}}}
\newcommand \Edk {{\xi_{dk}}}
\newcommand \AEdj {{\Lambda(\xi_{dj})}}
\newcommand \AEdk {{\Lambda(\xi_{dk})}}
\newcommand \AEd  {{ \bm{\Lambda}(\bm{\xi}_d) }}

\newcommand \Axi { { \Lambda_{\xi} } }
\newcommand \bxi { { \vv{b}_{\xi} } }
\newcommand \cxi { { c_{\xi} } }


\newcommand \wdoc      { { \vv{w}_d } }
\newcommand \wdt[0]  { { w_{dt} } }
\newcommand \wdn[0]  { { \vv{w}_{dn} } }
\newcommand \wdnt[0]  { { w_{dnt} } }
\newcommand \wdd[0]   { { \vv w_{d} } }
\newcommand \zd[0]   { { \vv z_{d} } }
\newcommand \zdn[0]  { { \vv{z}_{dn} } }
\newcommand \zdnk[0] { { z_{dnk} } }
\newcommand \zdk[0]  { { z_{dk} } }
\newcommand \thd[0]  { { \vv \theta_d } }
\newcommand \thdk[0] { { \theta_{dk} } }
\newcommand \thdj[0] { { \theta_{dj} } }
\newcommand \epow[1] { { e^{#1} } }
\newcommand \pkt     { { \phi_{kt}  } }
\newcommand \pk      { { \vv \phi_k } }
\newcommand \lmd     { { \vv \lambda_d } }
\newcommand \lmdk    { { \lambda_{dk} } }
\newcommand \xd      { { \vv x_d } }
\newcommand \atxd     { A ^\top \bm x_d}
\newcommand \axd     { A\bm x_d}
\newcommand \tsq      { { \tau^2 } }
\newcommand \ssq      { { \sigma^2 } }
\newcommand \tmsq     { { \tau^{-2} } }
\newcommand \asq      { { \alpha^2 } }
\newcommand \amsq     { { \alpha^{-2} } }
\newcommand \sgsq     { { \sigma^2 } }
\newcommand \xvec     { { \vv{x} } }
\newcommand \omk      { { \bm \omega _k } }
\newcommand \omkt     { { \omega_{kt} } }
\newcommand \oma     { { \Omega_A } }
\newcommand \gdn      { { \vv{\gamma}_{dn} } }
\newcommand \gdnk     { { \gamma_{dnk} } }
\newcommand \gdk      { { \gamma_{dk} } }
\newcommand \isigt   { { \Sigma^{-1}_{\bm \theta} } }




\newcommand \halfSig { \frac{1}{2\sigma^2} }

\newcommand \nor[2]   { \mathcal{N} \left( {#1}, {#2} \right) }
\newcommand \nord[3]   { \mathcal{N}_{#1} \left( {#2}, {#3} \right) }
\newcommand \mnor[3]  { \mathcal{N} \left(#1, #2, #3\right) }
\newcommand \norp[3]  { \mathcal{N} \left(#1; #2, #3\right) }
\newcommand \mnorp[4] { \mathcal{N} \left(#1; #2, #3, #4\right) }
\newcommand \mul[1]   { \mathcal{M} \left( {#1} \right) }
\newcommand \muln[2]  { \mathcal{M} \left( {#1},{#2} \right) }
\newcommand \dir[1]   { \mathcal{D} \left( {#1} \right) }
\newcommand \pois[1]  { \mathcal{P} \left( {#1} \right) }
\newcommand \gp[2]    { \mathcal{GP} \left( {#1}, #2 \right) }
\newcommand \dir[1]   { \mathcal{D} \left( {#1} \right) }
\newcommand \gam[2]   { \mathcal{G} \left( {#1}, {#2} \right) }
\newcommand \beta[1]  { \mathcal{B}eta \left( {#1}, {#2} \right) }

\newcommand \lne[1]  { { \ln \left( 1 + e^{ #1 } \right) } }
\newcommand \Tr[1]   { \tr \left(  {#1}  \right) }

\newcommand \roud  { \vv{\rho}_{d}  }
\newcommand \rodk { \rho_{dk} }

\newcommand \exA[1]  { \ex{#1}{q(A)} }
\newcommand \exV[1]  { \ex{#1}{q(V)} }
\newcommand \exT[1]  { \ex{#1}{q(\Theta)} }
\newcommand \extd[1] { \ex{#1}{q(\thd)} }
\newcommand \exTV[1] { \ex{#1}{q(\Theta)q(V)} }

\newcommand \Real[0]  { { \mathbb{R} } }
\newcommand \VReal[1] { { \mathbb{R}^{#1} } }
\newcommand \MReal[2] { { \mathbb{R}^{#1 \times #2} } }
\newcommand \Nat[0]  { { \mathbb{N} } }
\newcommand \VNat[1] { { \mathbb{N}^{#1} } }
\newcommand \MNat[2] { { \mathbb{N}^{#1 \times #2} } }

\newcommand \inv[1] { {#1}^{-1} }
\newcommand \invb[1] { \inv{\left( #1 \right)} }

\newcommand \cn { \textsuperscript{\texttt{[{\color{blue}Citation Needed}]}} }

\newcommand \const { { \text{c} } }

\providecommand \floor [1] { \left \lfloor #1 \right \rfloor }
\providecommand \ceil [1] { \left \lceil #1 \right \rceil }


\newcommand \vt[2] { { #1^{(#2)} } }

\newcommand \hashtag[1] { { \ttfamily \##1 } }

\newcommand \mvy  { \vv{m}_{\vv{y}} }
\newcommand \sigvy { { S_Y } }

\newcommand \mmy  { M_Y      }
\newcommand \md   { \vv{m}_d }
\newcommand \phin { \vv{\phi}_n }
\newcommand \isigma { { \inv{\Sigma} } }

\newcommand \sigv     { { \Sigma_V } }
\newcommand \isigv     { { \Sigma^{-1}_V } }

\newcommand \sigy { { \Sigma_Y } }
\newcommand \isigy { { \Sigma_{-1}_Y } }


\newcommand \omy  { { \Omega_Y } }
\newcommand \iomy { { \inv{\Omega_Y} } }

\newcommand \siga     { { \Sigma_A } }
\newcommand \isiga     { { \Sigma^{-1}_A } }
\newcommand \diagv { { \diag{\nu_1,\ldots,\nu_P} } }

\newcommand \ma { \vv{m}_a }
\newcommand \my { \vv{m}_y }

\newcommand \VoU { V \otimes U }

\newcommand \one { \mathbb{1} }
%\newcommand \one  {{  \mathds{1} }}

\newcommand \lse { \text{lse} }
%\newcommand \lse[0] { \mathrm{lse} }

% Conditional independence 
\def\ci{\perp\!\!\!\perp} % from Wikipedia



% ------ For the eval section

% Multinomial PDF [Math Mode]
% params: 1 - the variable
%         2 - the value
%         3 - the state indicator (e.g. k for a distro with K values)
%         4 - any additional subscript
\newcommand{\mpdf}[4] {
	\prod_{#3} {#1}_{{#4} {#3}} ^ {#2}
}

% Dirichlet PDF [Math Mode]
% params: 1 - the variable
%         2 - the hyper-parameter
%         3 - the state indicator (e.g. k for a distro with K values)
%         4 - any additional subscript
\newcommand{\dpdf}[4] {
	\frac{1}{B({#2})} \prod_{#3} {#1}_{{#4} {#3}} ^ {({#2}_{#3} - 1)}
}

% To simplify the sampling equations, this is indicates
% that the given value has had datapoint "m" stripped out
%
\newcommand{\lm}[1] {
	#1^{\setminus m}
}

\newcommand \model[0] {
    \mathcal{M}
}

\newcommand \perplexity[1] {
    \mathcal{P} \left( { #1 } \right)
}

\newcommand \WTrain {
    \mathcal{W}^{(t)}
}

\newcommand \WQuery {
    \mathcal{W}^{(q)}
}

\newcommand \oneover[1] {
    \frac{1}{ {#1} }
}

\newcommand \samp[1] {
    { #1 }^{(s)}
}

\newcommand \etd[0] {
    \vv{\eta}_d
}

\begin{document}



\subsection{Determining Appropriate Numbers of Mixture Components}
\label{sec:DPs}
In both mixture and admixture models, a problem is how to determine the appropriate number of components, $K$. The simplest possible method is to train a model on a number of different values of $K$, and then evaluate the model on held out data using some evaluation metric, perhaps using a heuristic such as canopy-clustering\cite{McCallum2000} to determine an appropriate range of values of $K$ in advance.

%In Bayesian inference, one can use priors to encode a preference for $K$. For example, if one employs a Dirichlet prior with a very small symmetric hyperparameter such $\alpha = 10^{-3}$ this encodes a preference for sparse models, where extra redundant clusters may never be used. 

Consider the two-parameter definition of a Dirichlet distribution (used in \cite{MacKay1995}\cite{Wallach2006}\cite{Wallach2009a} among others), where one writes $\vv{\theta} \sim \dir{\alpha \vv{m}}$ where $\vv{m}$ is the ``base-measure" on the simplex, and $\alpha$ is a ``concentration" parameter. This reflects the fact that two Dirichlet distributions may represent the distribution (e.g. $(2, 6, 10, 4)$ and $(0.2, 0.6, 1, 0.4)$) which differ only in ``concentration", i.e. by how much will a single observation affect the posterior. The manner in which the prior ``pseudo-counts" are transformed by the likelihood is sometimes described as ``Polya Urn" model.

%Thus the difference between a very small and very large symmetric hyperprior (where $\vv{m}=\vv{1}$) is only in the scale of the \emph{concentration parameter} $\alpha$, with small concentrations encoding ap reference 

To infer an appropriate number of mixture components automatically from the data, one can replace Dirichlet prior over mixture components with a Dirichlet process, which can be thought of as a the limit of a Dirichlet distribution over $K$ topics as $K \rightarrow \infty$. Formally, the Dirichlet Process is a distribution over distributions $G$, parameterised by a concentration parameter $\alpha$ and a base-measure $G_0$, such that
\begin{align*}
G \sim \text{DP}\left(\alpha, G_0\right)
\end{align*}
In order to describe the Dirichlet process it is useful to consider to related processes used for inference with DPs: the Chinese-Restaurant Process (CRP) and and the Stick-Breaking Process (SBP).

In the CRP\cite{Neal2000} a sequence of customers arrive to a restaurant, and for each customer, the restaurant can sit them at a newly prepared table, or else sit them a an existing table alongside other customers sat there previously. If we define a concentration parameter $\alpha$, the number of customers sat at a table $k$ as $n_k$ and the total number of customers as $n = \sum_k n_k$, then we can define the probability that the $(n+1)$-th customer will be sat at a new table as $\frac{1}{\alpha + 1}$ and the probability that a customer will be sat at table $k$ as $\frac{n_k + 1}{N + \alpha}$. 

For large values of $\alpha$ each customer is more likely to be sat at a new table, while for small values of $\alpha$ it is more likely that each customer will be sat at an existing table with other customers, which encodes a preference for fewer tables overall. The probability that a customer will be sat at a new table is never zero. Further, the more customers that are sat at a table, the more likely a new customer will also be sat at that table, encoding a ``rich get richer" property.

Therefore the CRP is a distribution over partitions of $n$ datapoints. Additionally, each  partition (i.e. table) can be assigned distribution $\phi_k \sim G_0$ drawn from a prior $G_0$, called ``base distribution". Note that many tables may have the same distribution (a metaphorical ``meal") assigned to them.

The stick-breaking process (SBP) handles the case of a CRP run without stop, i.e. for an infinite number of customers. The problem the stick-breaking process solves is determining which proportion of customers overall are sat at each table in the Chinese restaurant. The process starts by defining a ``stick" of length 1. A point is chosen between $0$ and $1$ where the stick should be broken, equivalent to making a draw $\pi_1 \sim \mathcal{B}eta\left(1, \alpha\right)$. The leftmost stick fragment is the first proportion. For the second proportion we break the remaining rightmost fragment, equivalent to making another draw $\pi_2 \sim \mathcal{B}eta\left(1, \alpha\right)$, and then rescaling it by the rightmost stick's length $1-\pi_1$. This can be applied recursively to generate an infinite number of stick subdivisions:
\begin{align}
\hat{\pi}_k & \sim \mathcal{B}eta\left(1, \alpha\right) & \pi_k = \hat{\pi}_k \prod_{j=1}^{k-1} (1 - \pi_j)
\end{align}

Since our stick was of length 1 to begin with, we know that $\sum_{k=1}^{\infty} \pi_k = 1$ and so it defines a valid probability distribution. From the properties of the Beta distribution, we know that for small values of $\alpha$ the initial stick breaks will be quite large, and so that most of the customers will be sat at the first few tables, whereas for large values, small breaks will be created so fewer customers will be sat at more tables.

With this we can begin to describe the Dirichlet Process. For the purposes of inference, one can use CRPs or SBPs to implement draws from DP priors. For example, suppose we want to generate a sample $\wdoc$ from $G$. One approach would be to implement a CRP\cite{Neal2000}, where each $\wdoc$ to arrive at the restaurant is sat at a table $k$ according to the CRP with concentration parameter $\alpha$, and each table is associated with some ``atom" $\phi_k \sim G_0$.

Equally, one could use a SBP\cite{Sethuraman1994} to generate weights $\pi_k$ with associated atoms $\phi_k$ and assign $\phi_k$ to observation $\wdoc$ with probability $\pi_k$. In practice, the total number of weights is generally truncated at some upper limit less than or equal to the size of the corpus. 

As should be clear from the presentation, the atoms associated with each table or stick-fragment can correspond to distributions, such as component distributions in a mixture model. The CRP naturally lends itself to the design of Gibbs sampling algorithms, whereas the SBP is typically used optimisation algorithms. 

A non-parametric equivalent to LDA called the Hierarchical Dirichlet Process (HDP) was derived in \cite{Teh2006b} using DPs. The model is specified as
\begin{align}
G_0 | \gamma, H & \sim \DP{\gamma, H } &
G_d | \alpha, G_0 & \sim \DP{\alpha, G_0} \\
\phi_{dn} |G_d &\sim G_d & w_{dn} | \phi_{dn} & \sim F(\phi_{dn})
\end{align}

Despite the superficial similarities to LDA, it should be noted that the whereas in LDA the parameters $\phi_{dn}$ are keys into a matrix of per-component probability-distributions, in HDP $\phi_{dn}$ is an atom from which a distribution over words is formed: specifically the base measure $H$ is the prior over \emph{words} (or other discrete observations), and takes the place of the $\dir{\vv{\beta}}$ prior over component distributions in LDA. Implementations using CRPs and SBPs are presented in\cite{Teh2006b}, while an in-depth description of DPs, CRPs, SBPs and their applications is given in\cite{JordanMichael2005a}

In practice, the distributions of topics often has a ``long" tail, there are few topics are very likely and many topics are less likely\cite{Wallach2009a}. In such cases one may make sense to use a Pitman-Yor Process, which captures, a power-law distribution over topics. Inference is achieved using a CRP augmented with a discount parameter, and this method has delivered better results than a HDP in experiments.\cite{Buntine2014}. However it is dependent on the dataset, and typically visible only when large numbers of topics are employed. 

Dirichlet processes, like Dirichlet distributions assume that components are near-independent. The Discrete Infinite Logistic Normal (DILN) distribution\cite{Paisley2012}\cite{Paisley2012b} incorporates a HDP into CTM in order to capture correlations amongst a potentially infinite number of components. The derivation proceeds in several stages. 
Firstly, one should note that one can implement a HDP using a combination of stick-breaking processes and gamma-processes, the latter analogous to the usual approach for sampling from a (finite) Dirichlet
%\begin{align}
%G_0 & = \sum_{k=1}^\infty \hat{\pi}_k \prod_{j=1}^{k-1} (1 - \hat{\pi}_j)\delta_{\phi_k} & 
%\hat{\pi}_k & \sim \mathcal{B}eta\left(1, \gamma\right) &
%\delta_{\phi_k} & \sim H \\
%G_d|G_0, Z & = \sum_{k=1}^\infty \frac{Z_k^{(d)}}{\sum_{j=1}^\infty Z_j^{(d)}} \delta_{\phi_k} &
%Z_k^{(d)} | G_0 &\sim \gam{\beta \pi_k}{1} &
%\pi_k & = \hat{\pi}_k \prod_{j=1}^{k-1} (1 - \hat{\pi}_j)
%\end{align}

To capture a sense of correlation, each atom $\phi_k$ is associated with a location $\vv{l}_k \in \VReal{d}$, where $d$ is set equal to $K$ for the purposes of inference
\begin{align}
G_0 | \gamma, H, L & \sim \DP{\gamma H \times L } &
G_d^{\text{DP}} | \alpha, G_0 & \sim \DP{\alpha, G_0} \\
\eta_d(l) & \sim \GP{\vv{m}(l)}{K(l, l')} &
G_d\left(\{\phi, l\}\right) & \propto G_d^{\text{DP}}\left(\{\phi, l\}\right) \exp\left(\eta_d(l) \right) \\
\phi_{dn} |G_d &\sim G_d & w_{dn} | \phi_{dn} & \sim F(\phi_{dn})
\end{align}

Noting that if $y = bx$ for $x \sim \gam{a}{1}$, then $y \sim \gam{a}{b^{-1}}$, we see we can fold in the exponentiated draw from the GP into the Gamma Process, and thus arrive at this model.

\begin{align}
\hat{\pi}_k & \sim \mathcal{B}\left(1, \gamma\right) &
\delta_{\phi_k} & \sim H,\quad \delta_{l_k} \sim L \\
G_0 & = \sum_{k=1}^\infty \hat{\pi}_k \prod_{j=1}^{k-1} (1 - \hat{\pi}_j)\delta_{\phi_k} & 
\pi_k & = \hat{\pi}_k \prod_{j=1}^{k-1} (1 - \hat{\pi}_j) \\
G_d|G_0, Z & = \sum_{k=1}^\infty \frac{Z_k^{(d)}}{\sum_{j=1}^\infty Z_j^{(d)}} \delta_{\phi_k, l_k} &
Z_k^{(d)} | G_0 &\sim \gam{\beta \pi_k}{e^{-\eta_{dk}}} \\
\eta_{dk} & \sim \GP{\vv{m}}{K} \\
\phi_{dn} |G_d &\sim G_d & w_{dn} | \phi_{dn} & \sim F(\phi_{dn})
\end{align}

A variational inference procedure is given in \cite{Paisley2012b} where a truncated Dirichlet was used to model the DP. Rather than learn a kernel function, the kernel (i.e. covariance) matrix was learnt directly from the values of $\eta_{dk}$ using the usual form $K = \frac{1}{D} \sum_d (\vv{\mu}_d - \vv{m})(\vv{\mu}_d - \vv{m}) + \diag{\vv{v}_d}$ where the variational posterior over the location for document d was $q(\vv{\eta}_d) = \nor{\vv{\mu}}{\diag{\vv{v}_d}}$. With regard to variational posterior over location, its parameters are found using gradient descent

%, where the gradients are
%\begin{align}
%\frac{\delta \mathcal{L}}{\delta \mu_{dk}} & = \lambda_{dk} - \beta \pi_k - K^{-1}_{k,:} \left( \vv{\mu}_d - \vv{m}\right) &
%\frac{\delta \mathcal{L}}{\delta v_{dk}} & = -\half \left( \lambda_{dk} - K^{-1}_{k,k} + \oneover{v_{dk}}\right)
%\end{align}
%and $\lambda_{dk} = \ex{Z_{dk}}{q} \times \ex{\exp(-\eta_{dk})}{q}$, and these expectations follow the usual form for gamma and log-normal distributions respectively. 

A variant of this approach was additionally used in \cite{Virtanen2012a}



%The nHDP is a generalization of the nested Chinese restaurant process (nCRP) that allows each word to follow its own path to a topic node according to a document-specific distribution on a shared tree. This alleviates the rigid, single-path formulation of the nCRP \cite{Paisley2012a}
%Data-dependent CRP relaxes exchangeability assumption\cite{Kim2011}

\bibliographystyle{plain}
\bibliography{/Users/bryanfeeney/Documents/library.bib}

\end{document}


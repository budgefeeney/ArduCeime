\documentclass[10pt,fleqn]{report}

\usepackage[english]{babel}
\usepackage[utf8x]{inputenc}
\usepackage{enumerate}
\usepackage{amsmath}
\usepackage{amssymb}
\usepackage{amsfonts}
\usepackage{mathtools}
\usepackage{graphicx}
\usepackage{bm}
\usepackage[usenames,dvipsnames]{color}
\usepackage{todonotes}
\usepackage{dsfont}
\usepackage{hyperref}
\hypersetup{
    colorlinks,
    citecolor=black,
    filecolor=black,
    linkcolor=black,
    urlcolor=black
}
\usepackage{algorithm}
\usepackage{algorithmic}
\usepackage{appendix}
\usepackage{caption}
\usepackage{subcaption}
\usepackage{fancyvrb}
\usepackage{subfigure}
\usepackage{graphicx,xcolor}
\usepackage{pifont,mdframed}
\usepackage{tikz}
\usetikzlibrary{fit,positioning}
\usepackage{setspace}
\onehalfspacing


\newcommand{\specialcell}[2][c]{%
  \begin{tabular}[#1]{@{}c@{}}#2\end{tabular}}

\newcommand \etr[0] {
    \text{etr}
}

\newcommand \Etr[1] {
    \text{etr}\left( { #1 } \right)
}

\newcommand \tinymath[1] {{  \mbox{\tiny ${#1}$ }  }}

%
% Macros
%
\newcommand \cashort[1] { {\todo[color=yello]{#1 -- Cedric}} }
\newcommand \calong[1]  { { \todo[inline,color=yellow]{#1 -- Cedric} } }
\newcommand \gbshort[1] { {\todo[color=cyan!40]{#1 -- Guillaume}} }
\newcommand \gblong[1]  { { \todo[inline, color=cyan!40]{#1 -- Guillaume} } }
\newcommand \mgshort[1] { {\todo{#1 -- Mark}} }
\newcommand \mglong[1]  { { \todo[inline]{#1 -- Mark} } }
\newcommand \bfshort[1] { {\todo[color=green!40]{#1 -- Bryan}} }
\newcommand \bflong[1]  { { \todo[inline,color=green!40]{#1 -- Bryan} } }


% Adds a plus const to the end of a math expression
\def \pcst{+\text{const}}

% A fancy version for capital R
\def \Rcal{\mathcal{R}}

% A fancy version for r
\def \rcal{\mathbf{r}}

% Loss function / log likelihood as appropriate
\def \L{\mathcal{L}}

% KL divergence [Math Mode]
\newcommand{\kl}[2] {
	\text{KL}\left[#1||#2\right]
}

\newcommand{\symmkl}[2] {
	\text{KL}^{symm}\left[#1||#2\right]
}

\newcommand \vecf[1] {
    \text{vec}\left(#1\right)
}

\newcommand \ent[1] {
    \text{H} \left[ #1 \right]
}

\newcommand \mut[2] {
    \text{I} \left[ #1 ; #2 \right]
}

\newcommand \dvi[2] {
    \text{D}_\text{VI} \left[ #1; #2 \right]
}

% Starts an expected value expresses [Math Mode]
\newcommand{\starte}[1] {%
	\mathbb{E}_{#1}\left[
}

% Ends an expected value expression [Math Mode]
\def \ende{\right]}

% Starts an varianc expresses [Math Mode]
\newcommand{\startv}[1] {%
	\mathbb{V}\text{ar}_{#1}\left[
}

% Ends an variance expression [Math Mode]
\def \endv{\right]}

%\newcommand \ex[2] {
%    \bigl\langle #1 \bigr\rangle_{#2}
%}
\newcommand \ex[2] {
    \mathbb{E}_{ { #2 } }\left[ #1 \right]
}
\newcommand \var[2] {
    \mathbb{V}\text{ar}_{ { #2 } }\left[ #1 \right]
}
\newcommand \cov[1] {
    \mathbb{C}\text{ov}[ #1 ]
}

\newcommand \halve[1] {
	\frac{#1}{2}
}

\newcommand \half {
    \halve{1}
}

\newcommand \tr { \text{tr} }

\newcommand \T { ^\top }

\newcommand \fixme[1] {
    {\color{red} FIXME: #1}
}

\newcommand \vv[1] { \bm #1 }

\newcommand{\mbeq}{\overset{!}{=}}

\def\app#1#2{%
  \mathrel{%
    \setbox0=\hbox{$#1\sim$}%
    \setbox2=\hbox{%
      \rlap{\hbox{$#1\propto$}}%
      \lower1.1\ht0\box0%
    }%
    \raise0.25\ht2\box2%
  }%
}
\def\approxpropto{\mathpalette\app\relax}

\newcommand \diag[1] { \text{diag} \left( {#1} \right) }
\newcommand \diagonal[1] { \text{diagonal} \left( {#1} \right) }

\newcommand \Ed {{ \vv{\xi}_d}}
\newcommand \Edj {{\xi_{dj}}}
\newcommand \Edk {{\xi_{dk}}}
\newcommand \AEdj {{\Lambda(\xi_{dj})}}
\newcommand \AEdk {{\Lambda(\xi_{dk})}}
\newcommand \AEd  {{ \bm{\Lambda}(\bm{\xi}_d) }}

\newcommand \Axi { { \Lambda_{\xi} } }
\newcommand \bxi { { \vv{b}_{\xi} } }
\newcommand \cxi { { c_{\xi} } }


\newcommand \wdoc      { { \vv{w}_d } }
\newcommand \wdt[0]  { { w_{dt} } }
\newcommand \wdn[0]  { { \vv{w}_{dn} } }
\newcommand \wdnt[0]  { { w_{dnt} } }
\newcommand \wdd[0]   { { \vv w_{d} } }
\newcommand \zd[0]   { { \vv z_{d} } }
\newcommand \zdn[0]  { { \vv{z}_{dn} } }
\newcommand \zdnk[0] { { z_{dnk} } }
\newcommand \zdk[0]  { { z_{dk} } }
\newcommand \thd[0]  { { \vv \theta_d } }
\newcommand \thdk[0] { { \theta_{dk} } }
\newcommand \thdj[0] { { \theta_{dj} } }
\newcommand \epow[1] { { e^{#1} } }
\newcommand \pkt     { { \phi_{kt}  } }
\newcommand \pk      { { \vv \phi_k } }
\newcommand \lmd     { { \vv \lambda_d } }
\newcommand \lmdk    { { \lambda_{dk} } }
\newcommand \xd      { { \vv x_d } }
\newcommand \atxd     { A ^\top \bm x_d}
\newcommand \axd     { A\bm x_d}
\newcommand \tsq      { { \tau^2 } }
\newcommand \ssq      { { \sigma^2 } }
\newcommand \tmsq     { { \tau^{-2} } }
\newcommand \asq      { { \alpha^2 } }
\newcommand \amsq     { { \alpha^{-2} } }
\newcommand \sgsq     { { \sigma^2 } }
\newcommand \xvec     { { \vv{x} } }
\newcommand \omk      { { \bm \omega _k } }
\newcommand \omkt     { { \omega_{kt} } }
\newcommand \oma     { { \Omega_A } }
\newcommand \gdn      { { \vv{\gamma}_{dn} } }
\newcommand \gdnk     { { \gamma_{dnk} } }
\newcommand \gdk      { { \gamma_{dk} } }
\newcommand \isigt   { { \Sigma^{-1}_{\bm \theta} } }

\newcommand \nd { { n_{d\cdot\cdot} } }


\newcommand \halfSig { \frac{1}{2\sigma^2} }

\newcommand \cat[1]   { \mathcal{C} \left( {#1} \right) }
\newcommand \nor[2]   { \mathcal{N} \left( {#1}, {#2} \right) }
\newcommand \nord[3]   { \mathcal{N}_{#1} \left( {#2}, {#3} \right) }
\newcommand \mnor[3]  { \mathcal{N} \left(#1, #2, #3\right) }
\newcommand \norp[3]  { \mathcal{N} \left(#1; #2, #3\right) }
\newcommand \mnorp[4] { \mathcal{N} \left(#1; #2, #3, #4\right) }
\newcommand \mul[1]   { \mathcal{M} \left( {#1} \right) }
\newcommand \muln[2]  { \mathcal{M} \left( {#1},{#2} \right) }
\newcommand \dir[1]   { \mathcal{D} \left( {#1} \right) }
\newcommand \pois[1]  { \mathcal{P} \left( {#1} \right) }
\newcommand \gp[2]    { \mathcal{GP} \left( {#1}, #2 \right) }
\newcommand \dir[1]   { \mathcal{D} \left( {#1} \right) }
\newcommand \gam[2]   { \Gamma \left( {#1}, {#2} \right) }
\newcommand \betadist[2]  { Beta \left( {#1}, {#2} \right) }
\newcommand \DP[1]    { \text{DP} \left( {#1} \right) }
\newcommand \PYP[1]   { \text{PYP} \left( {#1} \right) }
\newcommand \GP[2]    { \text{GP} \left( {#1}, {#2} \right) }

\newcommand \lne[1]  { { \ln \left( 1 + e^{ #1 } \right) } }
\newcommand \Tr[1]   { \tr \left(  {#1}  \right) }

\newcommand \roud  { \vv{\rho}_{d}  }
\newcommand \rodk { \rho_{dk} }

\newcommand \exA[1]  { \ex{#1}{q(A)} }
\newcommand \exV[1]  { \ex{#1}{q(V)} }
\newcommand \exT[1]  { \ex{#1}{q(\Theta)} }
\newcommand \extd[1] { \ex{#1}{q(\thd)} }
\newcommand \exTV[1] { \ex{#1}{q(\Theta)q(V)} }

\newcommand \Real[0]  { { \mathbb{R} } }
\newcommand \VReal[1] { { \mathbb{R}^{#1} } }
\newcommand \MReal[2] { { \mathbb{R}^{#1 \times #2} } }
\newcommand \Nat[0]  { { \mathbb{N} } }
\newcommand \VNat[1] { { \mathbb{N}^{#1} } }
\newcommand \MNat[2] { { \mathbb{N}^{#1 \times #2} } }

\newcommand \inv[1] { {#1}^{-1} }
\newcommand \invb[1] { \inv{\left( #1 \right)} }

\newcommand \cn { \textsuperscript{\texttt{[{\color{blue}Citation Needed}]}} }

\newcommand \const { { \text{c} } }

\providecommand \floor [1] { \left \lfloor #1 \right \rfloor }
\providecommand \ceil [1] { \left \lceil #1 \right \rceil }


\newcommand \vt[2] { { #1^{(#2)} } }

\newcommand \hashtag[1] { { \ttfamily \##1 } }

\newcommand \mvy  { \vv{m}_{\vv{y}} }
\newcommand \sigvy { { S_Y } }

\newcommand \mmy  { M_Y      }
\newcommand \md   { \vv{m}_d }
\newcommand \phin { \vv{\phi}_n }
\newcommand \isigma { { \inv{\Sigma} } }

\newcommand \sigv     { { \Sigma_V } }
\newcommand \isigv     { { \Sigma^{-1}_V } }

\newcommand \sigy { { \Sigma_Y } }
\newcommand \isigy { { \Sigma_{-1}_Y } }


\newcommand \omy  { { \Omega_Y } }
\newcommand \iomy { { \inv{\Omega_Y} } }

\newcommand \siga     { { \Sigma_A } }
\newcommand \isiga     { { \Sigma^{-1}_A } }
\newcommand \diagv { { \diag{\nu_1,\ldots,\nu_P} } }

\newcommand \ma { \vv{m}_a }
\newcommand \my { \vv{m}_y }

\newcommand \VoU { V \otimes U }

%\newcommand \one { \mathbb{1} }
\newcommand \one  {{  \mathds{1} }}

\newcommand \lse { \text{lse} }
%\newcommand \lse[0] { \mathrm{lse} }

% Conditional independence
\def\ci{\perp\!\!\!\perp} % from Wikipedia



% ------ For the eval section

% Multinomial PDF [Math Mode]
% params: 1 - the variable
%         2 - the value
%         3 - the state indicator (e.g. k for a distro with K values)
%         4 - any additional subscript
\newcommand{\mpdf}[4] {
	\prod_{#3} {#1}_{{#4} {#3}} ^ {#2}
}

% Dirichlet PDF [Math Mode]
% params: 1 - the variable
%         2 - the hyper-parameter
%         3 - the state indicator (e.g. k for a distro with K values)
%         4 - any additional subscript
\newcommand{\dpdf}[4] {
	\frac{1}{B({#2})} \prod_{#3} {#1}_{{#4} {#3}} ^ {({#2}_{#3} - 1)}
}

% To simplify the sampling equations, this is indicates
% that the given value has had datapoint "m" stripped out
%
\newcommand{\lm}[1] {
	#1^{\setminus m}
}

\newcommand \model[0] {
    \mathcal{M}
}

\newcommand \perplexity[1] {
    \mathcal{P} \left( { #1 } \right)
}

\newcommand \WTrain {
    \mathcal{W}^{(t)}
}

\newcommand \WQuery {
    \mathcal{W}^{(q)}
}

\newcommand \oneover[1] {
    \frac{1}{ {#1} }
}

\newcommand \samp[1] {
    { #1 }^{(s)}
}

\newcommand \etd[0] {
    \vv{\eta}_d
}

\begin{document}



\subsection{Evaluation Metrics}
\label{sec:eval}

A number of methods have been proposed to evaluate admixture models. By far the most popular is perplexity, which is defined as the geometric mean of the reciprocals of token likelihoods. Assuming exchangeability of words, the perplexity for a model $\model$ is defined as

\begin{align}
\perplexity{\model} = \sqrt[n_{d}]{\prod_d \prod_n \frac{1}{p(w_{dn}|\model)}} = \exp \left( \frac{-\sum_d \sum_n \ln p(w_{dn}|\model)}{n_{d\cdot\cdot}} \right) \label{eqn:perp_def}
\end{align}

As explained n \cite{Goodman2001} one can consider that a simple model that assigned uniform probability to 100 words would have a perplexity score of 100. The actual distribution, being non-uniform would be significantly \emph{lower}. Perplexity can also be linked to information theory: the log of a perplexity score gives the entropy of the word likelihoods, and as the entropy (base 2) specifies the average number of bits required to encode each word according to Shannon's source-coding theorem, lower perplexity values indicate better compression schemes.

Taking this further, perplexity can be shown to be a special case of the cross-entropy function $\text{H}[p,q]$ where $p(w_{dv})$ is the empirical distribution of the word, $q(w_{dv})$ is the distribution we've inferred (confusingly denoted $p(w_{dn})$ in \eqref{eqn:perp_def}), and we use $e$ as the base and the natural log in the exponent. 

To evaluate perplexity, one needs to determine the probability of a query (i.e. test) set $\WQuery$ given the training set $\WTrain$. Consider an LDA model in which the document-level topic distributions $\thd$ have been marginalised out, leaving a Polya distribution over the latent per-token topic-assignments $\zdn$, parameterised by a prior $\vv{\alpha}$.

\begin{align}
p(\WQuery|\WTrain) = \int p(\WQuery|\Phi, \vv{\alpha})p(\Phi,\vv{\alpha} | \WTrain) d\Phi d\vv{\alpha}
\end{align}

where due to the assumed exchangeability of documents $p(\WQuery|\Phi, \vv{\alpha}) = \prod_ p(\vv{w}_d^{q}|\Phi, \vv{\alpha})$. Leaving aside the first term in the integrand, one can approximate this integral either by substituting in point estimates for $\Phi$ and $\vv{\alpha}$ or sampling from the posterior - the latter being particularly easy in cases where the model parameters have been fit using Gibbs sampling.

The issue that remains however is how to evaluate $p(\vv{w}_d|\Phi, \vv{\alpha})$, which is dependant on the latent variables $\zdnk$. There are a number of ways in which this can be evaluated, of which two have been used frequently in the literature.

The harmonic mean method has been used in\cite{Griffiths2004}\cite{Griffiths2005}\cite{Wallach2006}, all of which performed inference using collapsed Gibbs samplers. The approach is defined as
\begin{align}
\oneover{p(w)} = \sum_z \frac{p(z|w)}{p(w|z)} \approx \oneover{S} \sum_s \oneover{p(w|\samp{z})}
=  \text{HM}\left[ \left\{ p(\vv{w}|\samp{\vv{z}}, \Phi \right\}_{s=1}^{S}  \right]
\end{align}
where HM is the Harmonic Mean and the approximation comes from implementing an importance sampler using the target $p(z|w)$ as the proposal distribution. Implementing this we can see that 

The sampling distribution over $\zdnk$ is given by

\begin{align}
p(z_{dn} = k | \vv{w}_d, \vv{z}_d^{\setminus n}, \Phi, \vv{\alpha}) \propto \phi_{k,{w_{dn}}}\frac{n_{dk\cdot} + \alpha_k}{n_{d\cdot\cdot} + \sum_j \alpha_j}
\end{align}

The second approach is the left-to-right method, which has been used in \cite{Mimno2011} and \cite{Mimno2012a}. By factorising the likelihood as

\begin{align}
p(\vv{w}|\Phi, \vv{\alpha}) & = \prod_n p(w_n | \vv{w}_{<n}, \Phi, \vv{\alpha}) \\
& = \prod_n \sum_{\vv{z}_{<n}} p(w_n, \vv{z}_{<n}| \vv{w}_{<n}, \Phi, \vv{\alpha})
\end{align}
it is possible to approximate the likelihood using a variant a sequential Monte-Carlo, where each particle $p$ iteratively draws samples of $z^{(p)}_{dn'}$ for $n' \in 1 \ldots n_{d\cdot\cdot}$, and uses these samples to evaluate the overall probability of the likelihood. The full algorithm is given in \cite{Wallach2009}.

When the harmonic mean, left-to-right and several other methods including importance sampling and annealed importance sampling where compared in \cite{Wallach2009} it was found that the harmonic mean method tended to ``wildly overestimate" the true likelihood, while the left-to-right method gave the most accurate estimate.

A method of evaluating perplexity which tests the model's predictive power is ``document-completion" which has been used in \cite{Virtanen2012a}\cite{Asuncion2012}\cite{RosenZvi2004} and measure's the model's predictive power. Each document in the test-set is partitioned into two halves\footnote{Given that words are assumed exchangeable within documents, and that long documents may exhibit structure affecting which topics can be inferred where, one can, and indeed should, permute the word order before splitting each document}: an estimation fragment, $\vv{w}_d^{(e)}$, from which a topic distribution is inferred; and an evaluation document, $\vv{w}_d^{(l)}$, whose likelihood given the inferred topic distribution is calculated. Formally writing this as
\begin{align}
p(\vv{w}_d^{(l)}|\vv{w}_d^{(e)}, \Phi, \vv{\alpha}) = \frac{p(\vv{w}_d^{(l)},\vv{w}_d^{(e)} |\Phi, \vv{\alpha})}{p(\vv{w}_d^{(e)}|, \Phi, \vv{\alpha})}
\end{align}

it should be clear that the methods used above can be used to infer both $p(\vv{w}_d^{(e)}|, \Phi, \vv{\alpha})$ and $p(\vv{w}_d^{(l)},\vv{w}_d^{(e)} | \Phi, \vv{\alpha}) = p(\vv{w}_d | \Phi, \vv{\alpha})$. However the most  popular method used in this case is topic-estimation which involves sampling $\vv{z}_d^{(e,s)} \sim p(\vv{z}_n^{(e)} | \vv{w}^{(e)}, \Phi, \vv{\alpha})$ and then evaluating $\vv{\theta}_d^{(e)}$ as

\begin{align}
\theta^{(e,s)}_{dk} = p(k | \vv{z}^{(e,s)}, \vv{\alpha}) = \frac{n^{(e,s)}_{dk\cdot} + \alpha_k}{n^{(e)}_{d\cdot\cdot} + \sum_j \alpha_j}
\end{align}

from which one can then evaluate the probability of the evaluation document:

\begin{align}
p(\vv{w}_d^{(l)}|\vv{w}_d^{(e)}, \Phi, \vv{\alpha}) \approx
\oneover{S} \sum_s \prod_n \sum_k \phi_{k,w^{(l)}_{dn}}\theta_{dk}^{(e,s)}
\end{align}

In practice, particularly with variational approaches, a single point estimate of $\vv{\theta}_d^{(e)}$ is often used instead\cite{Asuncion2012}, though when evaluating document completion \cite{Wallach2009} found that the topic-estimation method, even when implemented via sampling, compared ``poorly" with left-to-right sampling. 

Thus measures of perplexity are highly dependant on how the held-out likelihood is evaluated, which is rarely described well in the literature. Furthermore, many implementors tuse hard-coded symmetric priors for baseline topic models, even though the performance of topic models is highly sensitive to the configuration of hyper-parameters\cite{Asuncion2012}\cite{Wallach2006}. Consequently is is difficult to compare different measures of perplexity across different models in different publications.

Models with low perplexity scores are not always interpretable. In\cite{Chang2009} human testers were asked to inspect the topic vocabularies generated by different topic models (LDA, CTM and PLSA). In a ``word-intrusion" test they were asked which word, in a list of words belonging to a single topic, was the wrong word from another topic added by the experimenters. In a ``document-intrustion" they were similarly asked to identify the intruded document in a list of document excerpts belonging to a single topic. The CTM model, despite achieving the best perplexity score, faired worst in both of these tasks. 

%In   \cite{Azzopardi2003} it was shown that when using language-models to aid information retrieval, language models with the best perplexity scores did not always lead to the best precision scores. 

In light of this, some more recent publications such as \cite{Li2006}\cite{Wang2007}\cite{Lindsey2012} have used similar intrusion tests via Mechanical Turk to test the performance of their models.

There are metrics which don't rely on the likelihood. The ``topical coherence" metric, introduced in \cite{Mimno2011} and additionally used in \cite{Mimno2012a} addresses the word-intrusion problem. Assume that if a set of words all belong to a single topic, one would them concentrated in documents belonging to that topic, rather than distributed uniformly across all documents and all topics. Therefore letting $D(t)$ be the number of documents where word $t$ occurs at least once, and $D(t,v)$ be the number of documents containing at least one occurrence each of both $t$ and $v$, the coherence metric is defined as
\begin{align}
C(k; V^{(m)}) = \sum_{t \in V^{(m)}_k} \sum_{v \in V^{(m)}_k,  v \neq t} \ln \frac{D(t, v) + 1}{D(v)}
\end{align}
where $V^{(m)}_k$ is the set of the $m$ most probable words in topic $k$, and the addition of $1$ to the numerator is to avoid taking the log of zero. 

%\fixme{Read this some more. Also pointwise mutual info}

The \emph{empirical likelihood} of \cite{Li2006} tests the generative ability of a model. For each test document, this involves using the model to generate, say, 1,000 synthetic documents, deriving an empirical (in this case multinomial) distribution from each document, and then evaluating the average of the probabilities of the test document according to each of these empirical distributions. This has additionally been used in \cite{Doyle2009}\cite{Mimno2008}. In \cite{Wallach2009} it was shown that this method is an approximation to $p(\vv{w}_d|\Phi, \vv{\alpha}) = \int p(\vv{w}_d, \thd | \Phi)p(\thd|\vv{\alpha})d\thd$ using importance sampling with true posterior employed as a proposal distribution. Were the synthetic documents infinitely long, the two methods would be identical. This analogy points to a potential deficiency of this method however, which is that one would expect the variance of an estimate derived from importance sampling to be very high when sampling from high-dimensional distributions: consequently for very high topic-counts (cases with over 200-topic are common) this may not be a reliable estimator.

\subsection*{Evaluation Against Labelled Data}
With labelled data, specified with a 1-of-$L$ labelling $\vv{t}_d$, one may choose to evaluate the model by comparing it to the inferred topics, $\thd \in \VReal{K}$, where $K$ need not necessarily equal $L$. Many metrics exist for this purpose, of which the most well known is the F1-Score, $F_1 = 2 \frac{p \cdot r}{p + r}$, which is the harmonic mean of precision, $p=\frac{TP}{TP+FP}$ and recall, $r=\frac{TP}{TP+FN}$, and $TP, FP, TN, FN$ refer to the true and false positives and negatives respectively.

%In the presence of soft-clustering and multiple labels one can\footnote{See https://facwiki.cs.byu.edu/nlp/images/2/25/ClusterEvaluation.ppt} define an $F_1$ score for a single topic $k$ and a single label $l$ thusly.

Letting \emph{precision} be the ``number of documents in a cluster that belong there", and recall be be ``did all the documents that should be in this cluster make it in" we derive the following:
\begin{align}
p_{kl} & = \frac{n_{kl}}{n_k} = \frac{\sum_d \theta_{dk} t_{dl}}{\sum_d \theta_{dk}}  & r_{kl} & = \frac{n_{kl}}{n_l} = \frac{\sum_d \theta_{dk} t_{dl}}{\sum_d t_{dl}} \label{eqn:myrecall}
\end{align}

where $n_{kl}$ is the number of documents in cluster $k$ which have the label $l$ and $n_k$ is the number of documents in cluster $k$. We can, for a single topic $k$ and a single label $l$ specify the $F_1$ score as $F_1^{(k,l)} = 2 \frac{p_{kl} \cdot {r_{kl}}}{p_{kl} + {r_{kl}}}$

For a single label $l$ one can take the max across topics $F_1^{(l)} = \max_{k \in 1\ldots K} F_1^{(k,l)}$ and then finally describe the entire corpus using a weighted average of label specific $F_1$ scores, with the weights proportional to the number of documents with each label where $D' = \sum_d \sum_l t_{dl}$ is the total number of label instances, to account for the fact that there may be documents with more than one label, and so $D' \geq D$.


\begin{figure}
  \centering
    \includegraphics[width=0.9\textwidth]{plots/20news-2013-03-25.pdf}
  \caption{Results using different accuracy metrics for a mixture of multinomials and LDA on a subset of the 20-news dataset. Trendlines and shading are from a LOESS fit to the points. A variation of the soft F1 measure described in the text is shown where we average over label-specific F1 scores instead of taking the maximum. Held-out likelihood was evaluating using the document-completion metric with inferred topics.}
  \label{fig:eval-metrics-shootout}
\end{figure}


A different approach to comes from the intuition that given two equivalent labelings, one would expect label pairs $z=k$ and $t=l$ to co-occur, and so expect that $p(z,t) \neq p(z)p(t)$, which opens the door to a number of information theoretic measures such as normalised mutual information (NMI) and the Variation of Information (VI) distance..

\newcommand{\TopDist}{\Theta}
\newcommand{\LabDist}{\mathbb{C}}
\newcommand{\NMI}{\text{NMI}}

Given a corpus-wide distribution over topics $\TopDist$ and a corpus wide distribution over labels $\LabDist$ the NMI is defined as a function of the mutual informaiton $\mut{\TopDist}{\LabDist}$

\begin{align}
\NMI & = \frac{\mut{\TopDist}{\LabDist}}{1/2(\ent{\TopDist} + \ent{\LabDist})} &
\mut{\TopDist}{\LabDist} & = \sum_k \sum_l p(\theta_{\cdot k}, t_{\cdot l}) \ln \frac{p(\theta_{\cdot k}, t_{\cdot l})}{p(\theta_{\cdot k})p(t_{\cdot l})}
\end{align}
and the entropies are defined as
\begin{align}
\ent{\TopDist} & = - \sum_k p(\theta_{\cdot k}) \ln (\theta_{\cdot k}) &
\ent{\LabDist} & = - \sum_l p(t_{\cdot l}) \ln p(t_{\cdot l})
\end{align}

In the case of LDA these distributions would be defined as
\begin{align}
p(\theta_{\cdot k}) & =  \frac{1}{D} \sum_d \theta_{dk} ,&
p(t_{\cdot l}) &= \frac{1}{D'} \sum_d t_{dl} ,& 
p(\theta_{\cdot k}, t_{\cdot l}) & = \frac{1}{D'} \sum_d \theta_{dk} t_{dl}
\end{align}
where again $D' = \sum_d \sum_l t_{dl}$ is the total number of label instances, such that $D' \geq D$


The VI distance\cite{Meila2003} is another information-theoretic measure defined over a topic distribution $\TopDist$ and labelling distribution $\LabDist$ as:
\begin{equation}
\dvi{\TopDist}{\LabDist} = \ent{\LabDist} + \ent{\TopDist} - 2 \mut{\LabDist}{\TopDist}.
\end{equation}

$\dvi{\TopDist}{\LabDist}$ is a true metric: it is always non-negative, becomes zero if and only if $\LabDist = \TopDist$, is symmetric, and observes the triangle inequality, $\dvi{X}{Z} + \dvi{Z}{Y} \geq \dvi{X}{Y}$. However the range over which VI distance scores will take is dataset-dependant, and so to compare scores across datasets one can use the \emph{normalized VI distance}\cite{Reichart2009} which is defined as. 

\begin{align}
\text{NVI}\left[ \TopDist, \LabDist \right] = \left\{ \begin{array}{lr}
     \frac{1}{\ent{\TopDist}}\dvi{\TopDist}{\LabDist} & \ent{\TopDist} \neq 0 \\
     \ent{\LabDist} & \text{otherwise}
 \end{array}\right.
\end{align}


The VI distance has been used to compare topics inferred by LDA with manual labelling of over 20,000 news-stores in \cite{HeinrichEtAl2005}. 


A well-known metric from the field of clustering is the rand Rand Index, which determines how similar are two ways of partitioning data. It counts fours classes of document-pairs: (a) those which have the same cluster and have the same label; (b) those which have different clusters but share the same label; (c) those which have the same cluster but have different labels; and (d) those which which are not in the same cluster and have different labels.

Given these definitions the index is then $\text{RI} = \frac{a + d}{a + b + c + d}$. This is broadly similar to the idea of accuracy traditionally defined as being $\frac{TP + TN}{TP + FN + FP + TN}$

For the kind of soft-clustering featured in LDA, one would define these counts as
\begin{equation}
\begin{aligned}
a & = \sum_k \sum_{d, {p>d}} \theta_{dk} \theta_{pk} \sum_l t_{dl}t_{pl} & \quad
b & = \sum_k \sum_{d, {p>d}} \theta_{dk} (1 -\theta_{pk}) \sum_l t_{dl}t_{pl}\\
c & = \sum_k \sum_{d, {p>d}} \theta_{dk} \theta_{pk} \sum_{l,m \neq l} t_{dl}t_{ml} & \quad
d & = \sum_k \sum_{d, {p>d}} \theta_{dk} (1- \theta_{pk}) \sum_{l,m \neq l} t_{dl}t_{ml}
\end{aligned}
\end{equation}

Plots of all these metrics, for different topic counts, are given in figure \ref{fig:eval-metrics-shootout} for documents taken from four very different newsgroups from the overall 20-newsgroups dataset.


\bibliographystyle{plain}
\bibliography{/Users/bryanfeeney/Documents/library.bib}

\end{document}


